\documentclass{fkssolpub}

\usepackage[czech]{babel}
\usepackage{fontspec}
\usepackage{fkssugar}
\usepackage{amsmath}

\author{Ondřej Sedláček}
\school{Gymnázium Oty Pavla} 
\series{1-1}
\problem{4} 

\begin{document}

V mém řešení jako první pomůcky ze zadání si uložím do nafukovacích polí, kde
vždy budou pomůcky jednoho typu. Pak zkontroluji, jestli od každého typu je alespoň
jedna pomůcka, a případně vrátím 0, pokud se tato podmínka nesplnila.
Každé nafukovací pole pomůcek pak seřadím podle ceny v čase
$\mathcal{O}(n \log n)$ a následně každou nejlevnější pomůcku daného typu dám
do vyhledávacího stromu řadící podle kvality a zjistím jejich celkovou cenu.
Když už teď je celková cena vyšší než $m$, vrátíme 0. Pak
do té doby, kdy celková cena nepřesahuje $m$, najdeme pomůcku nejmenší kvality,
najdeme nejlevnější pomůcku stejného typu, ale vyšší kvality, a tuto pomůcku
přidáme do vyhledávacího stromu a aktualizujeme celkovou cenu. V případě, kdy
nenajdeme pomůcku vyšší kvality nebo kdy cena přesáhne $m$, vrátíme kvalitu
pomůcky nejmenší kvality. Protože můžeme až $n$-krát přidávat prvky do vyhledávacího
stromu, časová složitost této smyčky bude $\mathcal{O}(n \log t)$.

Celková časová složitost je proto $\mathcal{O}(n \log n + n \log t)$ a prostorová
složitost je $\mathcal{O}(n)$.

\end{document}
