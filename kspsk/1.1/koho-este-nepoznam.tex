\documentclass{fkssolpub}

\usepackage[czech]{babel}
\usepackage{fontspec}
\usepackage{fkssugar}
\usepackage{amsmath}

\author{Ondřej Sedláček}
\school{Gymnázium Oty Pavla} 
\series{1-1}
\problem{5} 

\begin{document}

Protože nám stačí jakkoliv zněmožnit únik FIITákům, ale Matfyzákům ho umožňit,
stačí každého jednotlivého FIITáka obestavět. Zeď ale na určité sousední políčko
FIITáka nebudeme dávat, když je kolem \verb|F|, nebo když je kolem \verb|M|.
V případě \verb|F| můžeme pokračovat, avšak v případě \verb|M| prohlásíme, že
řešení neexistuje. Obestavování bude trvat $\mathcal{O}(mn)$.

Jakmile obestavování dokončíme, spočítáme počet Matfyzáků na mapě a následně
spustíme BFS, abychom spočítali počet Matfyzáků, kteří můžou odejít z budovy.
Pokud můžou odejít všichni Matfyzáci, vrátíme řešení, jinak prohlásíme, že
řešení neexistuje.

Tohle všechno zvládneme s časovou složitostí $\mathcal{O}(mn)$ a s paměťovou
složitostí $\mathcal{O}(mn)$ na uložení, upravování mapy a pro BFS.

\end{document}
