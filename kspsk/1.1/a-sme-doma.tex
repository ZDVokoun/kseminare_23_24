\documentclass{fkssolpub}

\usepackage[czech]{babel}
\usepackage{fontspec}
\usepackage{fkssugar}
\usepackage{amsmath}

\author{Ondřej Sedláček}
\school{Gymnázium Oty Pavla} 
\series{1-1}
\problem{2} 

\begin{document}

Potom, co načtu všechna mysteriózní patra, pro zjednodušení výpočtu přidám k nim
ještě patra $z - 1$ a $k + 1$. Pak pro všechna patra najdeme největší vzdálenost
dvou sousedních pater, což uděláme tak, že všechna patra seřadíme podle jejich čísel
v čase $\mathcal{O}(p \log p)$ a pak postupně najdeme tu nejdelší vzdálenost v čase
$\mathcal{O}(p)$ a tu vrátíme. Celková časová složitost je proto $\mathcal{O}(p \log p)$
a prostorová složitost je $\mathcal{O}(p)$, protože pro seřazení musíme ty patra
načíst.

\end{document}
