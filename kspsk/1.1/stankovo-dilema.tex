\documentclass{fkssolpub}

\usepackage[czech]{babel}
\usepackage{fontspec}
\usepackage{fkssugar}
\usepackage{amsmath}

\author{Ondřej Sedláček}
\school{Gymnázium Oty Pavla} 
\series{1-1}
\problem{3} 

\begin{document}

Zde je důležité si všimnout toho, že aby $p_i \, \& \, p_{i+1} > p_i \oplus p_{i+1}$,
musí čísla $p_i$ a $p_{i+1}$ mít v bitové reprezentaci nejvýznamější bit na
stejném indexu. To zjistíme pro hodnotu $x$ tak, že zjistíme hodnotu
$\lfloor \log_2 x \rfloor$, což jsme schopni schopni spočítat v logaritmickém čase
vůči velikosti $x$. Pak v staticky alokovaném poli o velikosti
32 $A$ budeme ukládat počet předmětů, jejichž číslo má nejvýznamější bit v určitém
indexu. Výsledek pak bude:

\[
	\sum_{i=0}^{31} \binom{A_i}{2} = \sum_{i=0}^{31} \frac{A_i (A_i - 1)}{2}
\]

Protože takhle zjistíme počet možných kombinací. Protože nad každým číslem $p_i$
vypočítáme $\lfloor \log_2 p_i \rfloor$, bude celková složitost $\mathcal{O}(n \log p)$,
kde $p$ je maximální velikost čísla. A poněvadž alokujeme pořád stejně velké pole,
paměťová složitost bude konstantní při načteném vstupu.

\end{document}
