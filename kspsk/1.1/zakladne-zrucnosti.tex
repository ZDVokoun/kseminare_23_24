\documentclass{fkssolpub}

\usepackage[czech]{babel}
\usepackage{fontspec}
\usepackage{fkssugar}
\usepackage{amsmath}

\author{Ondřej Sedláček}
\school{Gymnázium Oty Pavla} 
\series{1-1}
\problem{1} 

\begin{document}

Mé řešení jako první vyřeší jediný edge case, a to případ, kdy $n < 3$. Tehdy musí
být výstup vždy 0. Pak v čase $\mathcal{O}(n)$ najdeme délku nejdelší tužky $m_1$ a
délku druhé nejdelší tužky $m_2$ tím, že porovnáme délku každé tužky s aktuálně uloženými
maximy. Pak vrátíme hodnotu výrazu $\min (m_2 - 1, n - 2)$, protože jednou počet
tužek určuje celkový počet tužek a jindy délka druhé nejdelší tužky. Jelikož tento
výpočet provedeme v čase $\mathcal{O}(1)$, celková časová složitost je $\mathcal{O}(n)$
a prostorová složitost $\mathcal{O}(1)$ při předpokladu, že vstup máme již načtený (jinak
samozřejmě je složitost lineární).

\end{document}
