\documentclass{fkssolpub}

\usepackage[czech]{babel}
\usepackage{fontspec}
\usepackage{fkssugar}
\usepackage{amsmath}

\author{Ondřej Sedláček}
\school{Gymnázium Oty Pavla} 
\series{1-2}
\problem{3} 

\begin{document}

Jako první potřebujeme zjistit komponenty souvislosti (místnosti) rozložení budovy.
To zjistíme snadno, z každého nenavštíveného bodu spustíme prohledávání (třeba
DFS) a dosažitelné políčka mřížky označíme stejným číslem v novém dvourozměrném
poli. To bude mít časovou i prostorovou složitost $\mathcal{O}(n m)$.

Následně z vytvořeného pole spočítáme obsahy jednotlivých místností a vytvoříme
nový graf, kde vrcholy budou místnosti a hrany mezi vrcholy značí, že spolu
sdílejí zeď. Nový graf získáme tím, že projdeme každé políčko a pokud
sousedí s políčkem jiné místnosti, přidáme mezi nimi hranu. Abychom však předešli duplicitním
hranám, budeme ukládat nový graf jako pole množin. Asymptoticky je nejlepší
použít hashtable, ale kvůli konstantě bych spíš použil implementace na základě
vyhledávacího stromu. A protože místnost může sousedit nejvíce s $\mathcal{O}(n + m)$
místnostmi, asymptotická složitost tvorby nového grafu je $\mathcal{O}(n m \log (n + m))$.

Pak musíme najít největší místnost vytvořenou spojením dvou místností. Pokud
jsme zjistili, že celou budovu tvoří jedna místnost, vrátíme obsah té místnosti.
Jinak zkoušíme sčítat obsahy každé dvojice místností, které jsou v námi vytvořeném grafu
spojeny hranou. To bude trvat $\mathcal{O}(n m)$, protože nemůže být v budově více
místností než $n m$. Pak již budeme znát výsledek, který můžeme vrátit.

Celková časová složitost je proto $\mathcal{O}(n m \log(n + m))$ a prostorová
složitost je $\mathcal{O}(n m)$.

\end{document}
