\documentclass{fkssolpub}

\usepackage[czech]{babel}
\usepackage{fontspec}
\usepackage{fkssugar}
\usepackage{amsmath}

\author{Ondřej Sedláček}
\school{Gymnázium Oty Pavla} 
\series{1-2}
\problem{2} 

\begin{document}

Jako první přiřadíme k symbolům jejich cifry, a to tak, že posloupnost symbolů
budeme procházet z leva doprava a budeme přiřazovat k symbolům cifry. To si pak
budeme ukládat v mapě (slovníku), nebo v poli konstantní velikosti, protože symboly
můžou být jen z omezeného rozpětí znaků. Symbol nejvyššího
řádu (nejvíce nalevo) bude cifra \verb|1| a druhý nejlevější symbol, který je
jiný od prvního, bude cifra \verb|0|, a každý další symbol bude nejmenší cifra,
která ještě nebyla použita. Tento
postup bude s použitím řazené mapy $\mathcal{O}(l \log l)$, ale s použitím
pole $\mathcal{O}(l)$.

Tímto postupem jsme zároveň schopni zjistit základ soustavy tohoto čísla.
Pokud je použita je jen jedna cifra, pak je číslo zapsaný v dvojkové soustavě,
jinak základ bude roven počtu různých cifer v čísle. Pak hodnotu
čísla vypočítáme jako geometrickou řadu:

\[
	\sum_{k=0}^{l} x_k \cdot r^k
\]

Kde $x_k$ je cifra na $k$-tém místě odzadu a $r$ je základ čísla. Výpočet pak
bude trvat, pokud budeme mít hodnoty symbolů uložené v poli, $\mathcal{O}(l)$.

Celkem je asymptotická složitost celého algoritmu $\mathcal{O}(t \cdot l)$, když
budeme místo slovníku používat pole konstantní velikosti. Protože pole bude
používat pořád stejně místa, je bez načteného vstupem prostorová složitost
$\mathcal{O}(1)$.

\end{document}
