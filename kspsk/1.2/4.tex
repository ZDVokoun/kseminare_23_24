\documentclass{fkssolpub}

\usepackage[czech]{babel}
\usepackage{fontspec}
\usepackage{fkssugar}
\usepackage{amsmath}

\author{Ondřej Sedláček}
\school{Gymnázium Oty Pavla} 
\series{1-2}
\problem{4} 

\begin{document}

Protože tohle je úloha, kde cíl je najít komponenty zadaného grafu, z každého
aktuálně nenavštíveného vrcholu budeme zpouštět DFS, který bude kopírovat
cestu, kterou voda stékat. Pak pokud při prohledávání narazíme na vrchol,
která je součástí již nalezené komponenty, pak všechny vrcholy prohledané
cesty jsou součástí též komponenty. Pokud však dojdeme do neoznačeného údolí, označíme
všechny vrcholy novým číslem. Tím pádem každý vrchol navštívíme nejvýše jednou.

Pokud takhle budeme označovat čísly od jedničky vrcholy zleva doprava a shora
dolů, splníme podmínku ze zadání. Pak uloženou mřížku s označenými vrcholy
vrátíme.

Celkově algoritmus bude mít časovou a prostorovou složitost $\mathcal{O}(n m)$.

\end{document}
