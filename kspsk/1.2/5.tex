\documentclass{fkssolpub}

\usepackage[czech]{babel}
\usepackage{fontspec}
\usepackage{fkssugar}
\usepackage{amsmath}

\author{Ondřej Sedláček}
\school{Gymnázium Oty Pavla} 
\series{1-2}
\problem{5} 

\begin{document}

Pro mé řešení je důležité následující pozorování. Nechť $i$ je index
určitého čísla a $j$, $k$ jsou indexy větších nebo stejně velkých čísel,
pro které platí $j < i < k$.
Pak počet úseků, které mají maximum číslo na indexu $i$, je
$(i - j) \cdot (k - i)$. Tohle pozorování platí podobně i pro minima úseků.

V mém algoritmu proto jako první si uložím vstup jako pole párů výšky a indexu
klasu a následně pole setřídím. Pak získám součet minim tím, že si vytvořím
vyhledávací strom s hodnotami $\{-1, n\}$ a pro každé číslo od nejmenšího
po největší zjistím pomocí vyhledávacího stromu najdu indexy $j$, $k$, které
jsem zmiňoval výše. Pak přičtu k celkovýmu součtu
$(i - j) \cdot (k - i) \cdot x_i$ a pokud $(i - j) \cdot (k - i) > 1$,
přidám index $i$ do vyhledávacího stromu, jinak je to zbytečné.
Jakmile pak projdeme všechny
prvky posloupnosti, provedeme podobný postup, jen budeme postupovat
od největšího k nejmenšímu, k získání součtu maxim.
Když už známe součet maxim a minim všech
posloupností, vrátíme rovnou supersumarizér.

Celý tento algoritmus má časovou složitost $\mathcal{O}(n \log n)$ a
prostorovou složitost $\mathcal{O}(n)$.


\end{document}
