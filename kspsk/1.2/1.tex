\documentclass{fkssolpub}

\usepackage[czech]{babel}
\usepackage{fontspec}
\usepackage{fkssugar}
\usepackage{amsmath}

\author{Ondřej Sedláček}
\school{Gymnázium Oty Pavla} 
\series{1-2}
\problem{1} 

\begin{document}

Abychom tuto úlohu vyřešili, stačí nám implementovat následující algoritmus --
pro každé políčko, které neleží v posledním sloupci či posledním řádku,
zkontrolujeme, jestli se jedná o levý horní roh diamantu. To určíme jen
na základě podmínky, že kontrolované políčko je \verb|/|, napravo od něj
je \verb|\|, dole od něj je \verb|\| a šikmo směrem doprava a dolů od něj
je \verb|/|. V průběhu si zaznamenáváme jejich počet, který pak nakonec vrátíme.

Protože pro každý políčko provedeme konstantně operací, časová složitost je
$\mathcal{O}(n m)$. Když budeme předpokládat, že pole už máme načtené, bude
prostorová složitost konstantní.

\end{document}
