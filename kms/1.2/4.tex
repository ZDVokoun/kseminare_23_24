\documentclass{fkssolpub}

\usepackage[czech]{babel}
\usepackage{fontspec}
\usepackage{fkssugar}
\usepackage{amsmath}

\author{Ondřej Sedláček}
\school{Gymnázium Oty Pavla} 
\series{1-2}
\problem{4} 

\begin{document}

Po zbytek řešení jsem stanovil bez újmy na obecnosti, že $|AD| = 1$, abych
si zkrátil zápis.

Protože jsou ramena $AD$ a $BC$ stejně dlouhá, musí se jednat o rovnoramenný
lichoběžník, proto úhly $|\angle DAB| = |\angle CBA| = \phi$ jsme schopni zjistit z pravoúhlého trojúhelníku
$AXD$, kde bod $X$ je pata výšky z bodu $D$ na stranu $AB$:

\[
	\cos \phi = \frac{|AX|}{|AD|} = \frac{\frac{|AB| - |CD|}{2}}{1} = \frac{1}{2}
\]
\[
	\phi = \arccos \frac{1}{2} = 60^{\circ}
\]

Zároveň pro výšku lichoběžníku $ABCD$ platí:

\[
	v = |AX| = \sin \phi = \frac{\sqrt{3}}{2}
\]

Čehož využiji pro výpočet obsah $S_{ABCD}$:

\[
	S_{ABCD} = \frac{1}{2} \cdot (|AB| + |CD|) \cdot v = \frac{5 \sqrt{3}}{4}
\]

Teď se pustíme do výpočtu obsahu trojúhelníku $ABE$. Protože si musíme nejprve
zkontrolovat, jestli se trojúhelník vejde do lichoběžníku, zjistíme nejprve
výšku trojúhelníku $ABE$:

\[
	\tan \frac{\phi}{2} = \frac{v_{ABE}}{\frac{|AB|}{2}}
\]
\[
	v_{ABE} = \frac{3}{2} \tan 30^{\circ} = \frac{\sqrt{3}}{2}
\]

Výšky lichoběžníku $ABCD$ a trojúhelníku $ABE$ jsou stejný, tudíž můžeme počítat
s celým obsahem toho trojúhelníku:

\[
	S_{ABE} = \frac{|AB| \cdot v_{ABE}}{2} = \frac{3 \sqrt{3}}{4}
\]

A zjistit část lichoběžníku $ABCD$ tvořenou trojúhelníkem $ABE$:

\[
	\frac{S_{ABE}}{S_{ABCD}} = \frac{\frac{3 \sqrt{3}}{4}}{\frac{5 \sqrt{3}}{4}}
	= \frac{3}{5}
\]

Trojúhelník $ABE$ zaplňuje 60\% obsahu lichoběžníku $ABCD$.

\end{document}
