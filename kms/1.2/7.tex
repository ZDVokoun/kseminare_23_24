\documentclass{fkssolpub}

\usepackage[czech]{babel}
\usepackage{fontspec}
\usepackage{fkssugar}
\usepackage{amsmath}

\author{Ondřej Sedláček}
\school{Gymnázium Oty Pavla} 
\series{1-2}
\problem{7} 

\begin{document}

Nejprve si ukážeme, že pro pole o sudé velikosti má vyhrávající strategii má druhý
hráč. Výherní strategie je jednoduchá -- opakovat tahy prvního hráče na středově
symetrických políčkách, než hrál první hráč, respektivě když první hráč hrál
na políčku se souřadnicemi $(i, j)$, bude druhý hráč hrát stejný tah na políčku
se souřadnicemi $(n + 1 - i, n + 1 - j)$. Druhý hráč nemůže pak udělat chybu,
protože touto strategií zařídí, aby vždy mohl hrát, pokud první hráč mohl odehrát.

Pro hrací pole o liché velikosti má ale výherní strategii první hráč. Důvod je
ten, že první hráč je schopný se dostat do stejné pozice jako druhý hráč v poli
o sudé velikosti tím, že zahraje jakýkoli tah do středu pole. Tím donutí druhého
hráče udělat tah a pak může první hráč vesele praktikovat strategii popsanou výše.


\end{document}
