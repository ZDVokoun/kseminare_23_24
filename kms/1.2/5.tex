\documentclass{fkssolpub}

\usepackage[czech]{babel}
\usepackage{fontspec}
\usepackage{fkssugar}
\usepackage{amsmath}

\author{Ondřej Sedláček}
\school{Gymnázium Oty Pavla} 
\series{1-2}
\problem{5} 

\begin{document}

Jako první rovnici ze zadání upravíme tak, abychom měli jen
jedno $x$, a to na levé straně:

\[
	\frac{x}{x + 4} = \frac{5 \lfloor x \rfloor - 7}{7 \lfloor x \rfloor - 5}
\]
\[
	x (7 \lfloor x \rfloor - 5) = (x + 4)(5 \lfloor x \rfloor - 7)
\]
\[
	x (2 \lfloor x \rfloor + 2) = 4 \cdot (5 \lfloor x \rfloor - 7)
\]
\[
	x (\lfloor x \rfloor + 1) = 10 \lfloor x \rfloor - 14
\]

Protože na levé straně musí být celé číslo, pro číslo $x$ musí platit
$x = \lfloor x \rfloor + \frac{k}{\lfloor x \rfloor + 1}$, kde číslo $k$
je celé nezáporné číslo a pro které $k < \lfloor x \rfloor + 1$. Po dosazení
jsme schopni získat tvar, u kterého můžeme rozebrat jednotlivé možnosti:

\[
	\left(\lfloor x \rfloor + \frac{k}{\lfloor x \rfloor + 1}\right) \left(\lfloor x \rfloor + 1\right)
	= 10 \lfloor x \rfloor - 14
\]
\[
	\lfloor x \rfloor^2 + \lfloor x \rfloor + k = 10 \lfloor x \rfloor - 14
\]
\[
	k = -\left(\lfloor x \rfloor^2 - 9 \lfloor x \rfloor + 14\right)
\]

Tento tvar teď upravíme na dva tvary. Jako první tu rovnici upravíme na:

\[
	k = -( \lfloor x \rfloor - 2) (\lfloor x \rfloor - 7)
\]

Z tohoto tvaru je vidět, že musí platit $\lfloor x \rfloor \in \langle 2; 7 \rangle$, aby
bylo $k$ nezáporné číslo. Ale teď tu rovnici upravíme na tento tvar:

\[
	-\left(\lfloor x \rfloor^2 - 9 \lfloor x \rfloor + (14 + k)\right) = 0
\]

Pomocí Vi\`{e}tových vzorců můžeme pak rozebrat všechny možné hodnoty
$\lfloor x \rfloor$ a $k$ ($b$ a $c$ značí koeficienty kvadratické rovnice
ve tvaru $x^2 + bx + c$):

\begin{table}[h!]
	\centering
	\begin{tabular}{|c|c|c|}
		\hline
		$-b$  & $c$              & $k$ \\
		\hline
		2 + 7 & $2 \cdot 7 = 14$ & 0   \\
		3 + 6 & $3 \cdot 6 = 18$ & 4   \\
		4 + 5 & $4 \cdot 5 = 20$ & 6   \\
		\hline
	\end{tabular}
	\caption{Rozbor možných výsledků pomocí Vi\`{e}tových vzorců}
\end{table}

Díky tabulce a podmínce $k < \lfloor x \rfloor + 1$ víme,
že $\lfloor x \rfloor \notin \{ 3; 4; 5\}$. Jediná řešení jsou proto:

\[
	x = 2 \qquad \lor \qquad x = 7 \qquad \lor \qquad x = 6 + \frac{4}{6 + 1} = \frac{46}{7}
\]

\end{document}
