\documentclass{fkssolpub}

\usepackage[czech]{babel}
\usepackage{fontspec}
\usepackage{fkssugar}
\usepackage{amsmath}

\author{Ondřej Sedláček}
\school{Gymnázium Oty Pavla} 
\series{1-2}
\problem{8} 

\begin{document}

Jako první formulujeme vzorec pro obsah lichoběžníku ze zadání $S(k)$.
Nejprve však musíme zjistit výšku tohoto lichoběžníku. Protože ze zadaných
hodnot vyplývá, že se musí jednat o rovnoramenný lichoběžník, výšku zjistíme
přes Pythagorovu větu:

\[
	v = \sqrt{3^2 - \frac{(k-3)^2}{4}} = \frac{1}{2} \sqrt{-k^2 + 6k + 27}
	= \frac{1}{2} \sqrt{-(k+3)(k-9)}
\]

Teď můžeme konečně formulovat vzorec pro obsah lichoběžníku:

\[
	S(k) = \frac{1}{4} (k + 3) \sqrt{-(k+3)(k-9)}
\]

Abychom zjistili maximum obsahu $S(k)$, musíme tento vzorec zderivovat:

\[
	S'(k) = \frac{(k - 6)(k + 3)}{2 \sqrt{-(k+3)(k-9)}}
\]

Protože extrémy funkce $S(k)$ jsou v bodech, kdy $S'(x) = 0$, jen v bodě 6 může
být extrém. Musíme si však ověřit, že tento bod je opravdu extrém a jestli se
jedná o maximum.

Prvně si uvědomíme, že funkce $S(k)$ má definiční obor $\langle-3, 9\rangle$.
Výraz $\sqrt{-(k+3)(k-9)}$ má zřejmě jenom tři extrémy, z nichž v bodech -3 a 9
je globální minima a zbývající je globální maximum. A
protože v definičním oboru bude výraz $k + 3$ vždy kladný, funkce $S(k)$ bude mít
stejně extrémů a stejný počet globálních maxim. A protože extrém, který jsme našli,
je v bodě 6, nikoliv v bodech -3 či 9, našli jsme globální maximum.

Proto maximálního obsahu dosáhneme, když $k = 6$, kdy pozemek bude mít obsah
$S(6) = \frac{27 \sqrt{3}}{4}$.

\end{document}
