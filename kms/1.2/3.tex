\documentclass{fkssolpub}

\usepackage[czech]{babel}
\usepackage{fontspec}
\usepackage{fkssugar}
\usepackage{amsmath}

\author{Ondřej Sedláček}
\school{Gymnázium Oty Pavla} 
\series{1-2}
\problem{3} 

\begin{document}

Jako první určíme, jak pro každou cifru se změní ciferný součet při vynásobení
dvěmi. Protože nemůže dojít k přetečení při přičtení carry (protože velikost
carry může být nejvýše jedna a nejvyšší sudé jednociferné číslo je 8), stačí
nám si vytvořit tabulku níže:


\begin{table}[h!]
	\centering
	\begin{tabular}{|c|c|c|c|c|c|}
		\hline
		Cifra & Po vynásobení & Změna cif. součtu \\
		\hline
		0     & 0             & 0                 \\
		1     & 2             & +1                \\
		2     & 4             & +2                \\
		3     & 6             & +3                \\
		4     & 8             & +4                \\
		5     & 1 + 0         & -4                \\
		6     & 1 + 2         & -3                \\
		7     & 1 + 4         & -2                \\
		8     & 1 + 6         & -1                \\
		9     & 1 + 8         & 0                 \\
		\hline
	\end{tabular}
	\caption{Rozbor cifer}
\end{table}

Jelikož číslo 18 není dvojnásobek čísla 12, musíme použít alespoň jedno z čísel
5, 6, 7, 8. Ale ať už vybereme jakékoli z těchto čísel, největší změna, které
jsme schopni dosáhnout, je +3. Protože však chceme získat změnu +6, abychom z
ciferného součtu 12 dostali 18, došli jsme ke sporu.

Tudíž Bezzubý si zřejmě ušetřil práci a na žádné číslo nemyslel.


\end{document}
