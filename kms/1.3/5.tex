\documentclass{fkssolpub}

\usepackage[czech]{babel}
\usepackage{fontspec}
\usepackage{fkssugar}
\usepackage{amsmath}

\author{Ondřej Sedláček}
\school{Gymnázium Oty Pavla} 
\series{1-3}
\problem{5} 

\begin{document} 

Označme horizontálně polarizovaný foton jako \verb|0| a vertikálně polarizovaný foton jako 
\verb|1|.

Nejprve prodiskutujme tuto úlohu pro hodnoty $n \in \{2, 3\}$. Pro $n = 2$ je jediná možná
sekvence fotonů, kdy jev nastane, je \verb|11|, a pro $n = 3$ je tato sekvence
\verb|011|. To se nám hodí, protože je z toho vidět, že pro sekvenci fotonů o $n > 3$
musí nutně platit, že sekvence posledních tří fotonů je \verb|011|. Tudíž nám stačí
jen získat způsob, jak určit pravděpodobnost, že po tom, co uvidí $n - 3$ fotonů, 
nezahlédne dva vertikálně polarizované fotony za sebou.

Pro určení tohoto vzorce vůči $k$ si nejdříve musíme uvědomit, že po horizontálně polarizovaném
fotonu může následovat jakýkoli foton, kdežto po vertikálně polarizovaném fotonu může
následovat jen horizontálně polarizovaný foton. Níže je schéma pro přehlednost:

\begin{verbatim}
0 -> 0, 1
1 -> 0
\end{verbatim}

Pak určíme funkční vztah posloupnosti $a_k$, určující celkový počet případů, 
kdy po $k$ fotonech neskončí, posloupnosti $h_k$, určující počet
případů, kdy posloupnost končí horizontálně polarizovaným fotonem, a posloupnosti 
$v_k$, určující počet případů, kdy posloupnost končí vertikálně polarizovaným fotonem:

\[
  a_k = h_k + v_k
\]
\[
  h_k = h_{k - 1} + v_{k - 1}
\]
\[
  v_k = h_{k - 1}
\]

Po dosazení $v_{k - 1}$ do vztahu pro $h_k$, získáme rekurentní vztah jak pro 
$h_k$, tak pro $v_k$:
\[
  h_k = h_{k - 1} + h_{k - 2}
\]
\[
  v_k = v_{k - 1} + v_{k - 2}
\]

Musíme však nejprve určit výchozí hodnoty těchto posloupností. Zřejmě platí 
$h_1 = 1$ a $v_1 = 1$, protože jsme teprve v první vrstvě rozhodovacího stromu a 
nemůže dojít k tomu, že by skončil po jednom fotonu. Pak víme, že $v_2 = h_1 = 1$
a $h_2 = h_1 + v_1 = 1 + 1 = 2$.

Tyto dvě posloupnosti jsou nápadně podobné Fibonacciho posloupnosti. Obě mají stejný
funkční vztah jako Fibonacciho posloupnost, a taky $v_1 = F_1$, $v_2 = F_2$, $h_1 = F_2$
a $h_2 = F_3$. Z toho vyplývá, že je můžeme vyjádřit pomocí prvků Fibonacciho posloupnosti:

\[
  v_k = F_k
\]
\[
  h_k = F_k + F_{k - 1} = F_{k + 1}
\]

Teď jsme schopni elegantně vyjádřit hodnotu $a_k$:

\[
  a_k = F_{k + 1} + F_k = F_{k + 2}
\]

Když už máme všechno potřebné, vyjádříme konečně pravděpodobnost toho, že po $n$ 
fotonech uvidí dva vertikálně polarizované fotony za sebou. Protože pravděpodobnost, 
že sekvence fotonů o velikosti 3 bude
\verb|011|, je $\frac{1}{8}$ a pravděpodobnost, že v sekvenci fotonů o velikosti
$n - 3$ nebudou žádné dva vertikálně polarizované fotony za sebou, je 
$\frac{a_{n - 3}}{2^{n - 3}}$, hledaná pravděpodobnost bude:

\[
  \frac{a_{n - 3}}{2^{n - 3}} \cdot \frac{1}{8} = \frac{F_{n - 3 + 2}}{2^n} 
    = \frac{F_{n - 1}}{2^n}
\]

Můžeme si všimnout, že tento vztah platí i pro $n \in \{1, 2, 3\}$, a to přestože jsme
pracovali s podmínkou $n > 3$. Proto tento vzorec platí pro všechna $n \in \mathbb{N}$.

Q. E. D.

\end{document}
