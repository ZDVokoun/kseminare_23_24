\documentclass{fkssolpub}

\usepackage[czech]{babel}
\usepackage{fontspec}
\usepackage{fkssugar}
\usepackage{amsmath}

\author{Ondřej Sedláček}
\school{Gymnázium Oty Pavla} 
\series{1-3}
\problem{8} 

\begin{document} 

Pro účely důkazu si definuju posloupnost prvočísel $p_n$, kde je ale $p_0 = 1$,
přestože se o prvočíslo nejedná. Zbavím se tím několika okrajových případů.

Když si zkusíme vypsat několik příkladů množství šošovek od začátku, můžeme
určit, že pro každé přirozené $n$ je možné množství šošovek $x p_n$, kdy 
přirozené číslo $x \in \langle p_{n-1}; p_{n+1})$ a $p_n$ je největší prvočíselný
dělitel.

Pro $n = 1$ to platí zřejmě, protože na začátku máme $1 \cdot 2$ šošovek, a při
přičítání dvojky postupně dostaneme $3 \cdot 2$, kdy je největší prvočíselný dělitel
$p_2 = 3$.

Pro jakékoli přirozené $n$ pak začínáme na $x = p_{n - 1}$. Přidáváním $p_n$ šošovek
se postupně dostaneme na $x = p_{n + 1}$, protože posloupnost prvočísel je rostoucí
a toto je nejmenší prvočíslo, které je ostře větší než $p_n$. Zřejmě nemůže být $x$
složené číslo, protože žádné složené číslo v tomto intervalu bude mít většího prvočíselného
dělitele než $p_n$, proto se při této hodnotě mění největší prvočíselný dělitel. Tím
jsme dokázali, co jsme vypozorovali.

Aby pro každé přirozené $n$ byl celkový počet šošovek čtverec, musí platit $x = k^2 p_n$ 
pro přirozené $k$. Víme, že $x < p_{n+1}$, musí tedy platit:

\[
  k^2 p_n < p_{n + 1}
\]

Protože podle Bertrandova postulátu platí $p_{n + 1} < 2p_n$ pro přirozené $n$, platí:

\[
  k^2 p_n < p_{n + 1} < 2p_n
\]

Odkud je zřejmé, že tato podmínka je splněna právě když $k = 1$.

Proto jediné hodnoty $m$, pro které může být na stole $m^2$ šošovek, jsou prvočíselné
a žádné jiné.

Q. E. D.

\end{document}
