\documentclass{fkssolpub}

\usepackage[czech]{babel}
\usepackage{fontspec}
\usepackage{fkssugar}
\usepackage{amsmath}

\author{Ondřej Sedláček}
\school{Gymnázium Oty Pavla} 
\series{1-3}
\problem{4} 

\begin{document} 

Pro vytvoření kolmice ze zadanéhého bodu $X$ vůči rovině $\rho$ jako první zkontruujeme
kouli se středem $X$, který má na povrchu nějaký bod roviny $\rho$. Pokud je průnik
kulové plochy této koule a roviny bod, tvoří rovina tečnou rovinu a stačí už jen 
sestrojit přímku sestrojením dvou rovin, na kterých leží jak tento průnik, tak zadaný 
bod $X$. Pokud však je průnik kružnice, vytvoříme tři koule s různými středy, které leží
na této kružnici a které mají na povrchu bod $X$. Když získáme průnik jejich kulových
ploch, dostaneme bod $X$ a bod $X'$. Bod $X'$ musí být vůči bodu $X$ osově souměrný,
kde osa je rovina $\rho$, protože známe tři body, od kterých jsou oba body stejně
vzdáleni. A protože přímka spojující tyto dva body tvoří kolmici vůči ose, stačí
zkontruovat přímky konstrukcí dvou různých rovin, na kterých leží oba body $X$ a
$X'$.

Pro zkontruování roviny souměrnosti si pomůžeme tím, že zkontruujeme rovinu kolmou
vůči oběma zadaným rovinám. Tu zkontruujeme tak, že si vybereme 2 body $X$ a $Y$ na průniku
obou rovin a zkontruujeme dvě koule, a to se středem $X$ a s bodem $Y$ na povrchu a 
u druhého naopak. Průnikem kulových ploch obou koulí dostaneme kružnici, která leží
na rovině kolmé vůči oběma zadaným rovinám, protože tato rovina tvoří osu souměrnosti
bodů $X$ a $Y$, a tím pádem obou zadaných rovin. Jakmile už máme tuto rovinu
zkontruovanou, můžeme převést tuto úlohu na planimetrickou tím, že budeme hledat
osu úhlu průniků kolmé roviny a zadaných rovin (neboli dvou přímek) v této rovině.
Potom rovinu souměrnosti zkontruujeme vytvořením roviny s bodem z osy úhlu a 
body $X$ a $Y$.

Pro čtyřstěn $ABCD$ musíme nejdříve pro konstrukci koule vepsané získat střed této
koule. Ten získáme tak, že postupně zkontruujeme roviny souměrnosti rovin $ABC$ a
$ABD$, rovin $ABC$ a $BCD$ a rovin $ABC$ a $ACD$. Jejich průnik pak bude bod $S$, který
bude od rovin každé stěny stejně vzdálen, tím pádem bude středem koule vepsané. Zbývá
nám pak získat její velikost, k čemuž nám stačí sestrojit kolmici z bodu $S$ na jednu
ze stěn, abychom získali její patu $P$. Koule vepsaná čtyřstěnu pak bude mít střed $S$
a na povrchu bude mít bod $P$.

\end{document}
