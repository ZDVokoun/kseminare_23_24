\documentclass{fkssolpub}

\usepackage[czech]{babel}
\usepackage{fontspec}
\usepackage{fkssugar}
\usepackage{amsmath}

\author{Ondřej Sedláček}
\school{Gymnázium Oty Pavla} 
\series{1.3}
\problem{6} 

\begin{document} 

Protože ta zárovka se součtem zlomků nám umožňuje jednoduše získat ekvivalentní
úpravou nerovnici s harmonickým průměrem, zkusíme zjistit, jaké hodnoty má aritmetický
průměr hodnot ve jmenovateli:

\[
  \frac{(b+2c+d) + (c+2d+a) + (d+2a+b) + (a+2b+c)}{4} = \frac{4a + 4b + 4c + 4d}{4}
    = a + b + c + d
\]

Z toho vyplývá, že nerovnice v zadání je jen upravená AH nerovnost (nerovnost mezi
aritmetickým a harmonickým průměrem), tudíž pro důkaz
stačí upravit nerovnici do tvaru běžnou pro AH nerovnost:

\[
  (a+b+c+d) \cdot \left(\frac{1}{b+2c+d} + \frac{1}{c+2d+a} + \frac{1}{d+2a+b} + \frac{1}{a+2b+c}\right) 
    \geq 4
\]
\[
  \frac{(b+2c+d) + (c+2d+a) + (d+2a+b) + (a+2b+c)}{4} \cdot 
    \left(\frac{1}{b+2c+d} + \frac{1}{c+2d+a} + \frac{1}{d+2a+b} + \frac{1}{a+2b+c}\right) 
    \geq 4
\]
\[
  \frac{(b+2c+d) + (c+2d+a) + (d+2a+b) + (a+2b+c)}{4}
    \geq \frac{4}{\frac{1}{b+2c+d} + \frac{1}{c+2d+a} + \frac{1}{d+2a+b} + \frac{1}{a+2b+c}}
\]

Tento tvar již zřejmě vyplývá z AH nerovnosti, která vychází z nerovnosti mezi
průměry.

\end{document}
