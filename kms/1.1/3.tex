\documentclass{fkssolpub}

\usepackage[czech]{babel}
\usepackage{fontspec}
\usepackage{fkssugar}

\author{Ondřej Sedláček}
\school{Gymnázium Oty Pavla} 
\series{1-1}
\problem{3} 

\begin{document}

Jako první zjistíme koeficient $k$ podobnosti trojúhelníků $DFI$ a $AED$. Ze zadání
jsme schopni získat tyto dvě rovnosti:

\[
	|AE| = |DF| + |FC|
\]
\[
	|DF| = 2 |FC|
\]

Z těchto rovností jsme schopni jednoduše zjistit $k$:

\[
	k = \frac{|DF|}{|AE|} = \frac{2}{3}
\]

Teď máme na výběr dva způsoby, jak zjistit výšku trojúhelníku $ABJ$. První,
který mě napadl, spočíval v tom, že když si představíme, že čtverců v krabici
je takhle naskládáno nekonečně mnoho, můžeme výšku trojúhelníku $ABJ$ vyjádřit
jako geometrickou řadu, protože postupně se ty krabice budou přibližovat bodu $J$:

\[
	|AB;J| = |AE| \cdot \sum^{\infty}_{n=1} \left(\frac{2}{3}\right)^{n-1}
	= \frac{1}{1 - \frac{2}{3}} \cdot |AE| = 3 |AE|
\]

Nebo to můžeme zjistit více geometrickým způsobem. Nechť je bod $X$ průsečík
přímky $AB$ a výšky trojúhelníku $ABJ$ z bodu $J$, tudíž úsečka $JX$ bude
tvořit výšku trojúhelníku $ABJ$ z bodu $J$. Pak jsou si trojúhelníky
$BXJ$ a $GCB$ podobné. Jednoduše pak můžeme zjistit, že $|CG| = \frac{1}{3}|AE|$,
díky čemuž jsme schopni získat soustavu rovnic:

\[
	\frac{|BX|}{|JX|} = \frac{1}{3}
\]
\[
	|JX| = |AB| + |BX| = 2|AE| + |BX|
\]

Z níž zjistíme, že:

\[
	3 |BX| = 2 |AE| + |BX| \ztoho |BX| = |AE| \ztoho |JX| = 3 |AE|
\]

Teď můžeme získat velikost části trojúhelníku $ABJ$ tvořena čtvercem $BCDE$:

\[
	\frac{|BCDE|}{|ABJ|} = \frac{|AE|^2}{\frac{2|AE| \cdot 3 |AE|}{2}} = \frac{1}{3}
\]

\end{document}
