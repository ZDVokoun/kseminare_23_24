\documentclass{fkssolpub}

\usepackage[czech]{babel}
\usepackage{fontspec}
\usepackage{fkssugar}
\usepackage{amsmath}

\author{Ondřej Sedláček}
\school{Gymnázium Oty Pavla} 
\series{1-1}
\problem{4} 

\begin{document}

Předpokládáme, že Bospor byl první na tahu. Pak Bospor zahrál celkem
jedenáctkrát. Proto nám stačí dokázat, že ať už obarvíme jakýkoliv
jedenáct políček na šachovnici $7 \times 3$, jsou čtyři takové, které
vytvoří obdélník podle zadání.

Nechť jsou řádky delší než sloupce, neboli šachovnice je tvořena 3 řádky o
velikosti 7 a 7 sloupci o velikosti 3. Pokud jeden z sloupci je celý vybarven, pak
pro vznik čtverce stačí, aby v alespoň jednom dalším sloupci byly alespoň
dvě vybarvená políčka. To podle Dirichletova principu musí platit, protože
zbývajících sloupců je šest, dalších políček k vybarvení je 8 a $8 > 6$.

Teď nám zbývá vyřešit případ, kdy není žádný sloupec vybarven celý. Pak musí
být minimálně 4 sloupce, kde jsou dvě vybarvená políčka. Ale protože
počet kombinací, jak vybarvit sloupec dvěmi barvami, jsou $\binom{3}{2} = 3$,
musí podle Dirichletova principu být jedno ze sloupců vybarveno stejně,
a tím pádem tvořit obdélník.

Protože jsme vyřešili všechny možné případy, je tímto důkaz u konce.


\end{document}
