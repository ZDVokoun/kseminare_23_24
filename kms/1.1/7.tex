\documentclass{fkssolpub}

\usepackage[czech]{babel}
\usepackage{fontspec}
\usepackage{fkssugar}

\author{Ondřej Sedláček}
\school{Gymnázium Oty Pavla} 
\series{1-1}
\problem{7} 

\begin{document}

Nechť polynom $Q(x) = P(x) - x$. Pak platí, že pro všechna reálná $x$ platí
$Q(x) \geq 0$, $Q(1) = 0$, $Q(2) = 2$, $Q(3) = 0$. Protože polynomy čtvrtého stupně
mají nejvýše tři extrémy, nemůže se jednat o konstantní funkci a platí
podmínka $Q(x) \geq 0$, musí mít polynom $Q(x)$ v hodnotách 1 a 3
dvojnásobné kořeny. Proto tento polynom musí být $Q(x) = k (x - 1)^2 (x - 3)^2$,
kde $k$ zjistíme z funkční hodnoty v bodě 2:

\[
	k \cdot (2 - 1)^2 \cdot (2 - 3)^2 = 2 \ztoho k = 2
\]

Z definice polynomu $Q(x)$ pak můžeme získat $P(x)$:

\[
	P(x) = 2 (x - 1)^2 (x - 3)^2 + x = 2 x^4 - 16 x^3 + 44 x^2 - 47 x + 18
\]

\end{document}
