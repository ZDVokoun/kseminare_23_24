\documentclass{fkssolpub}

\usepackage[czech]{babel}
\usepackage{fontspec}
\usepackage{fkssugar}
\usepackage{amsmath}

\author{Ondřej Sedláček}
\school{Gymnázium Oty Pavla} 
\series{1}
\problem{5} 

\begin{document}

Postup přelévaní, díky němuž získáme veškeré konfigurace, je
takový, že z první zkumavky budeme přelévat do druhé zkumavky
a pokud druhá zkumavka je plná, tak ji vylijeme a vylijeme zbytek
v první zkumavce (dále jako "přetečení").

Nechť je první zkumavka o velikosti $n$ a druhá o velikosti $k$.
Výše zmíněným způsobem prakticky aplikujeme modulární aritmetiku,
proto budeme samostatně počítat kroky, kdy normálně přeléváme a kdy
dojde k přetečení. Aby všechny konfigurace byly jedinečný, musíme
tento postup ukončit, jakmile přelijeme celkem $\text{lcm}(n,k)$
jednotek vody. Proto napustíme první zkumavku a
rovnou přelijeme do druhé celkem $\frac{\text{lcm}(n,k)}{k}$-krát.

Dále počet situací, kdy dojde k přetečení, je
$\frac{\text{lcm}(n,k)}{k} - 1$. Protože u každé této situace
provedeme dva kroky, počet kroků je:

\[
	2 \cdot \text{lcm}(n,k) \cdot \left(\frac{1}{n} + \frac{1}{k}\right) - 2
\]

Protože jsme ale nezahrnuli krok, kdy $(0, 0)$ se změní na $(n, 0)$ a
kdy se $(0, k)$ změní na $(n, k)$, konečný počet konfigurací je:

\[
	2 \cdot \text{lcm}(n,k) \cdot \left(\frac{1}{n} + \frac{1}{k}\right)
	= 2 \cdot \frac{n + k}{\text{gcd}(n,k)}
\]

\end{document}
