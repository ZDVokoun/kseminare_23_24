\documentclass{fkssolpub}

\usepackage[czech]{babel}
\usepackage{fontspec}
\usepackage{fkssugar}
\usepackage{amsmath}
\usepackage{graphicx}

\author{Ondřej Sedláček}
\school{Gymnázium Oty Pavla} 
\series{2-1}
\problem{4} 

\begin{document} 

Nejprve dokážeme, že $n \geq a(n)$. Víme, že každé $n$ můžeme zapsat
jako:

\[
  x_0 \cdot 10^0 + x_1 \cdot 10^1 + \dots + x_k \cdot 10^k \text{,}
\]

kde každé $x_i$ jsou cifry čísla. Protože pracujeme v desítkové
soustavě, nerovnost ze zadání můžeme zapsat jako:

\[
  x_0 \cdot 10^0 + x_1 \cdot 10^1 + \dots + x_k \cdot 10^k 
    \geq x_0 \cdot x_1 \cdots x_k
\]

Tuto nerovnost nebudeme řešit napřímo, protože jsme schopni jednoduše
dokázat, že:

\[
  x_k \cdot 10^k \geq x_0 \cdot x_1 \cdots x_k
\]

Víme, že pro každé $x_i$ platí $0 \geq x_i \geq 9$, tím pádem také vždy platí,
že $x_i < 10$. Z toho je už zřejmé, že nerovnost výše platí, protože na pravé
straně je jen $k + 1$ čitatelů.

A protože je zřejmé, že:

\[
  x_0 \cdot 10^0 + x_1 \cdot 10^1 + \dots + x_k \cdot 10^k \geq x_k \cdot 10^k
\]

Nerovnost $n \geq a(n)$ nutně platí.

Teď vyřešíme kvadratickou rovnici ze zadání. Víme, že musí platit nerovnost
$0 \leq a(n) \leq n$ (cifry jsou vždy nezáporný), a tedy získáme nerovnice:

\[
  n^2 - 17n + 56 \geq 0
\]
\[
  n^2 - 18n + 56 \leq 0
\]

Celočíselná řešení první nerovnice jsou 
$\mathbb{Z} \, \cap \, (- \infty; 4 \rangle \, \cap \, \langle 13; + \infty)$ a 
druhé nerovnice jsou $\mathbb{Z} \, \cap \, \langle 4 ; 14 \rangle$. Tím se
nám zúží počet možných řešení na \{4, 13, 14\}. Po vyzkoušení těchto hodnot
najdeme jen jediné řešení, a to $n = 4$.

Tím je řešení u konce.

\end{document}
