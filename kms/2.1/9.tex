\documentclass{fkssolpub}

\usepackage[czech]{babel}
\usepackage{fontspec}
\usepackage{fkssugar}
\usepackage{amsmath}
\usepackage{graphicx}

\author{Ondřej Sedláček}
\school{Gymnázium Oty Pavla} 
\series{2-1}
\problem{9} 

\begin{document} 

Můžeme si všimnout jedné zajímavé věci na této posloupnosti -- když
dosadíme $a_{n+1} = 2024 / a_n$, získáme:

\[
  a_{n+2} = \frac{a_n + 2024}{1 + \frac{2024}{a_n}} = a_n
\]

Tudíž když si vybereme taková celá $a_1$ a $a_2$, aby splňovali
podmínku výše, máme jistotu, že celá posloupnost bude z celých čísel.
To čísla $44 + 46 = 90$ splňují. Aby existovala menší taková hodnota,
muselo by v jednom okamžiku dojít k tomu, že by čísla celá čísla
$a_{n}$ a $a_{n+1}$ platilo $a_{n+1} = 2024 / a_n$. K tomu však nemůže
dojít, tudíž tohle je jediné řešení.

\end{document}
