\documentclass{fkssolpub}

\usepackage[czech]{babel}
\usepackage{fontspec}
\usepackage{fkssugar}
\usepackage{amsmath}
\usepackage{graphicx}

\author{Ondřej Sedláček}
\school{Gymnázium Oty Pavla} 
\series{2-1}
\problem{6} 

\begin{document} 

Z pozorování si můžeme všimnout, že po každém rozšíření je v posloupnosti
vždy jedna devítka a pak každé číslo přibyde v posloupnosti o tolik,
kolik bylo čísel větších než toto číslo před rozšířením. To můžeme
vyjádřit soustavou rekurentních vztahů (počítám prvky od jedničky):

\begin{gather*}
  a_n = 1 \\
  b_1 = 1 \qquad b_n = b_{n - 1} + a_{n - 1} \\
  c_1 = 1 \qquad c_n = c_{n - 1} + b_{n - 1} + a_{n - 1} \\
  d_1 = 1 \qquad d_n = d_{n - 1} + c_{n - 1} + b_{n - 1} + a_{n - 1} \\
  e_1 = 1 \qquad e_n = e_{n - 1} + d_{n - 1} + c_{n - 1} + b_{n - 1} + a_{n - 1} \\
  f_1 = 1 \qquad f_n = f_{n - 1} + e_{n - 1} + d_{n - 1} + c_{n - 1} + b_{n - 1} + a_{n - 1} \\
  g_1 = 1 \qquad g_n = g_{n - 1} + f_{n - 1} + e_{n - 1} + d_{n - 1} + c_{n - 1} + b_{n - 1} + a_{n - 1} \\
  h_1 = 1 \qquad h_n = h_{n - 1} + g_{n - 1} + f_{n - 1} + e_{n - 1} + d_{n - 1} + c_{n - 1} + b_{n - 1} + a_{n - 1} \\
  i_1 = 1 \qquad i_n = i_{n - 1} + h_{n - 1} + g_{n - 1} + f_{n - 1} + e_{n - 1} + d_{n - 1} + c_{n - 1} + b_{n - 1} + a_{n - 1} \\
\end{gather*}

Tyto vztahy lze pak zjednodušit na:

\begin{gather*}
  a_n = 1 \\
  b_1 = 1 \qquad b_n = b_{n - 1} + a_{n} \\
  c_1 = 1 \qquad c_n = c_{n - 1} + b_{n} \\
  d_1 = 1 \qquad d_n = d_{n - 1} + c_{n} \\
  e_1 = 1 \qquad e_n = e_{n - 1} + d_{n} \\
  f_1 = 1 \qquad f_n = f_{n - 1} + e_{n} \\
  g_1 = 1 \qquad g_n = g_{n - 1} + f_{n} \\
  h_1 = 1 \qquad h_n = h_{n - 1} + g_{n} \\
  i_1 = 1 \qquad i_n = i_{n - 1} + h_{n} \\
\end{gather*}

Tudíž pro získání explicitního vyjádření se zdá, že budeme muset vyřešit
spoustu sum. Pokud však prvky posloupností zapíšeme pomocí kombinačních
čísel, dost si zjednoduššíme práci:

\[
  b_n = \binom{0}{0} + \binom{1}{0} + ... + \binom{n - 1}{0} = \binom{n}{1}
\]
\[
  c_n = \binom{1}{1} + \binom{2}{1} + ... + \binom{n}{1} = \binom{n + 1}{2}
\]

Tudíž nás může napadnout, jestli dál $d_n = \binom{n + 2}{3}$ atd. Zobecněný
tvar tohoto vzorce dokážu indukcí, tedy budu chtít dokázat, že:

\[
  \sum_{i=1}^n \binom{i + k - 1}{k} = \binom{n + k}{k + 1}
\]

Pro první sumy už jsem to ukázal, takže nám zbývá dokázat indukční krok.
Nejprve si však sumu nalevo zapíšu vzestupně:

\[
  \binom{k}{k} + \binom{k + 1}{k} + \binom{k + 2}{k} + ... + \binom{n + k - 1}{k}
\]

Teď dosadíme $\binom{k}{k} = \binom{k + 1}{k + 1}$. Protože obecně pro
kombinační čísla platí $\binom{n}{k} + \binom{n}{k + 1} = \binom{n + 1}{k + 1}$
(vztah, který umožňuje sestrojení Pascalova trojúhelníku), za součet prvních
dvou členů můžeme dosadit $\binom{k + 2}{k + 1}$:

\[
  \binom{k + 2}{k + 1} + \binom{k + 2}{k} + \binom{k + 3}{k} + ... + \binom{n + k - 1}{k}
\]

Můžeme tedy pozorovat, že tento vztah můžeme používat opakovaně do té
doby, dokud nezískáme konečnou sumu, tedy platí:

\[
  \sum_{i=1}^n \binom{i + k - 1}{k} = \binom{n + k - 1}{k + 1} + \binom{n + k - 1}{k}
    = \binom{n + k}{k + 1}
\]

Což jsme chtěli dokázat. Teď pro počet jedniček potřebujeme získat správné
$k$. Protože jsme pro devítku používali $k = 0$, pak tedy pro počet jedniček
je $k = 8$, proto:

\[
  i_n = \binom{n + 7}{8}
\]

A protože tato posloupnost začíná od jedničky, počet jedniček po $n$-tém
rozšíření je:

\[
  \binom{n + 8}{8}
\]

Z rekurzivního vztahu pro počet jedniček zároveň víme, že délka celé
posloupnosti po $n$-tém rozšíření je stejný jako počet jedniček
po $n + 1$-tém rozšíření, a proto délka celé posloupnosti bude:

\[
  \binom{n + 9}{8}
\]

Tím jsme získali vše, co jsme chtěli.

\end{document}
