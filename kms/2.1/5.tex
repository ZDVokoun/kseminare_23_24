\documentclass{fkssolpub}

\usepackage[czech]{babel}
\usepackage{fontspec}
\usepackage{fkssugar}
\usepackage{amsmath}
\usepackage{graphicx}

\author{Ondřej Sedláček}
\school{Gymnázium Oty Pavla} 
\series{2-1}
\problem{5} 

\begin{document} 

Je zřejmé, že pokud $a^2 + k \mid (a - 1) a (a + 1)$, pak platí:

\[
  (a - 1) a (a + 1) \equiv 0 \pmod{a^2 + k}
\]

Tuto kongruenci můžeme dále upravit:

\[
  (a - 1) a (a + 1) \equiv (a - 1) (a^2 + a) \equiv (a - 1) (a - k) \equiv
  a^2 + k - ak - a \equiv -(ak + a) \equiv 0 \pmod {a^2 + k}
\]

Tudíž musí zároveň platit, že $a^2 + k \mid ak + a$. Protože čísla
$a$, $k$ jsou kladná celá čísla, musí platit:

\[
  ak + a \geq a^2 + k
\]

Jinak by podmínka $a^2 + k \mid ak + a$ nemohla platit. Nerovnost následně 
upravíme:

\[
  ak - k \geq a^2 - a
\]
\[
  (a - 1) k \geq (a - 1) a
\]

Poněvadž v případě $a = 1$ dokazovaná nerovnost $k \geq a$ bude vždy splněna,
můžeme pokrátit a dostáváme:

\[
  k \geq a
\]

Což jsme chtěli dokázat. Tím je důkaz u konce.

\end{document}
