\documentclass{fkssolpub}

\usepackage[czech]{babel}
\usepackage{fontspec}
\usepackage{fkssugar}
\usepackage{amsmath}
\usepackage{graphicx}

\author{Ondřej Sedláček}
\school{Gymnázium Oty Pavla} 
\series{FO-B-S}
\problem{4} 

\begin{document} 

V části a díky přímé úměrnosti tlaku na oběmu při ději 1-2 můžeme
vztah stavových veličin v bodech 1, 2 takto:

\[
  k p_1 = p_2
\]
\[
  k V_1 = V_2
\]

A z toho:

\[
  p_2 V_2 = n R T_2 = k^2 p_1 v_1 = k^2 n R T_1 \implies T_2 = k^2 T_1
\]

Teď sestavíme zbytek soustavy:

\[
  V_2 = V_3
\]
\[
  p_1 V_1 = p_3 V_3
\]

Práce vykonaná tímto cyklem je:

\[
  W_a = (V_2 - V_1) \frac{p_1 + p_2}{2} + (V_1 - V_3) \frac{p_1 + p_3}{2}
\]

Postupným dosazovaním rovnic výše se dostaneme ke chtěnému výsledku:

\begin{gather*}
  W_a = (V_2 - V_1) \frac{p_1 + p_2 - p_1 - p_3}{2} 
      = \frac{1}{2} V_1 (k - 1) (p_2 - p_3) 
      = \frac{1}{2} V_1 (k - 1) \left(k p_1 - \frac{p_1 V_1}{V_3}\right)
      = \frac{1}{2} p_1 V_1 (k - 1) \left(k - \frac{V_1}{V_2}\right) \\
      = \frac{1}{2} p_1 V_1 (k - 1) \left(k - \frac{1}{k}\right)
      = \frac{1}{2} p_1 V_1 \left(k^2 - k - 1 + \frac{1}{k}\right)
\end{gather*}

Protože teploty v bodech 1 a 2 známe, za $k$ je dosadíme a získáme:

\[
  W_a = \frac{1}{2} n R T_1 \left(\frac{T_2}{T_1} - \sqrt{\frac{T_2}{T_1}} + \sqrt{\frac{T_1}{T_2}} - 1\right)
\]

V části b to je podobné, jen přibyde rovnic:

\begin{gather*}
  k p_1 = p_3 \\
  k V_1 = V_3 \\
  k^2 T_1 = T_3 \\
  V_1 = V_2 \\
  V_3 = V_4 \\
  p_2 = p_3 \\
  p_1 V_1 = p_4 V_4 \\
\end{gather*}

A práci vypočítáme též podobným způsobem:

\begin{gather*}
  W_b = (V_4 - V_1) \left(p_2 - \frac{p_1 + p_4}{2}\right) 
      = \frac{1}{2} (V_3 - V_1) \left(2p_3 - p_1 - \frac{p_1 V_1}{V_4} \right) \\
      = \frac{1}{2} V_1 (k - 1) \left(2k p_1 - p_1 - \frac{p_1 V_1}{V_3} \right)
      = \frac{1}{2} p_1 V_1 (k - 1) \left(2k - 1 - \frac{1}{k} \right)
      = \frac{1}{2} p_1 V_1 \left(2k^2 - 3k + \frac{1}{k} \right)
\end{gather*}

A teď zase upravíme do požadovaného tvaru:

\[
  W_b = \frac{1}{2} n R T_1 \left(2 \frac{T_3}{T_1} - 3 \sqrt{\frac{T_3}{T_1}} + \sqrt{\frac{T_1}{T_3}} \right)
\]


\end{document}
