\documentclass{fkssolpub}

\usepackage[czech]{babel}
\usepackage{fontspec}
\usepackage{fkssugar}
\usepackage{amsmath}
\usepackage{graphicx}

\author{Ondřej Sedláček}
\school{Gymnázium Oty Pavla} 
\series{FO-B-S}
\problem{7} 

\begin{document} 

Protože je sestava kola na hřídeli složena ze dvou válců se stejnou
osou otáčení, pro zjištění momentu setrvačnosti $J$ sestavy sečteme
momenty setrvačnosti jednotlivých válců. Nejprve však musíme zjistit,
jak je rozdělena hmotnost mezi kolem a hřídelem:

\[
  \frac{m_k}{m_h} = \frac{V_k}{V_h} = \frac{\pi (2r)^2 \cdot r}{\pi r^2 \cdot 3r}
    = \frac{4}{3}
\]

Tím pádem moment setrvačnosti sestavy bude:

\[
  J = \frac{1}{2} \cdot \frac{4}{7} M \cdot (2r)^2 + \frac{1}{2} \cdot \frac{3}{7} M \cdot r^2
    = \frac{1}{2} M r^2 \cdot \frac{16 + 3}{7} = \frac{19}{14} M r^2
\]

Pro získání velikosti zrychlení $a$ formulujeme soustavu rovnic
skládající se z druhé impulzové věty a závislosti úhlového zrychlení
na zrychlení tělesa:

\[
  J \epsilon = F_{G0} \cdot 2r
\]
\[
  a = \epsilon \cdot 2r
\]

Když první rovnici dosadíme do druhé, získáme výsledek:

\[
  a = \epsilon \cdot 2r = \frac{F_{G0} \cdot 2r}{J} \cdot 2r
    = \frac{4 m_0 g \cdot r^2}{J} = \frac{56 m_0 g}{19 M}
\]

Pro další část úlohy musíme zjistit, jaké závaží bylo zavěšené na hřídeli.
To zjistíme z rovnováhy momentů sil:

\[
  m_0 g \cdot 2r = m_1 g \cdot r \implies m_1 = 2 m_0
\]

Teď formulujeme podobnou soustavu rovnic jako v předchozí části (jako
kladný směr uvažuji takový směr, že závaží na kole zrychluje dolů):

\[
  J \epsilon = 2 m_0 g \cdot 2r - m_0 g \cdot r = 3 m_0 g r
\]

\[
  a_k = \epsilon \cdot 2r
\]

\[
  a_h = \epsilon \cdot r
\]

Oba zrychlení získáme analogicky jako v předchozí části:

\[
  a_k = \frac{3 m_0 g r}{J} \cdot 2r = \frac{6 m_0 g \cdot r^2}{J} 
      = \frac{84 m_0 g}{19 M}
\]
\[
  a_h = \frac{3 m_0 g r}{J} \cdot r = \frac{3 m_0 g \cdot r^2}{J}
      = \frac{42 m_0 g}{19 M}
\]

Tím je řešení u konce.

\end{document}
