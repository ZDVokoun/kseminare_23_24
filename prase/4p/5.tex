\documentclass{fkssolpub}

\usepackage[czech]{babel}
\usepackage{fontspec}
\usepackage{fkssugar}
\usepackage{amsmath}

\author{Ondřej Sedláček}
\school{Gymnázium Oty Pavla} 
\series{4p}
\problem{5} 

\begin{document}

Let $a$, $b$, $c > 0$ be real numbers. From AG inequality we know that:

\[
	\frac{a + 2b + 3c}{6} \geq \sqrt[6]{ab^2c^3}
\]

Now suppose that $x = a$, $y = 2b$, $z = 3c$ holds. Then we can get the
value of the right side of the inequality:

\[
	xy^2z^3 = 108
\]
\[
	a (2b)^2 (3c)^3 = 108
\]
\[
	ab^2c^3 = 1
\]

So to get the smallest possible value of $x + y + z$, both sides of the inequality
has to be equal. Hence, the numbers $a$, $b$, $c$ has to be equal. As a result,
$a = b = c = 1$ and the smallest possible value of $x + y + z$ is:

\[
	x + y + z = a + 2b + 3c = 1 + 2 + 3 = 6
\]


\end{document}
