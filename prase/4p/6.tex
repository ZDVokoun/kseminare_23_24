\documentclass{fkssolpub}

\usepackage[czech]{babel}
\usepackage{fontspec}
\usepackage{fkssugar}
\usepackage{amsmath}

\author{Ondřej Sedláček}
\school{Gymnázium Oty Pavla} 
\series{4p}
\problem{6} 

\begin{document}

To start with, I will rewrite the inequality to this form:

\[
	\left| \sum_{k = 1}^n O(k) - \sum_{k = 1}^n E(k) \right| \leq n
\]

The reason why I rewrote the inequality like this is because we
can easily get the number of numbers from 1 to $n$ divisible by $k$
with formula $\left\lfloor n/k \right\rfloor$. As a result,
the inequality is equivalent to:

\[
	\left| \sum_{k = 1}^{\left\lceil n/2 \right\rceil} \left\lfloor \frac{n}{2k - 1} \right\rfloor
	- \sum_{k = 1}^{\left\lceil n/2 \right\rceil} \left\lfloor \frac{n}{2k} \right\rfloor \right| \leq n
\]

Now we will solve this inequality for two cases -- when the value in the vertical bars
are positive and when it is negative.

When the value is positive, then:

\[
	\sum_{k = 1}^{\left\lceil n/2 \right\rceil} \left\lfloor \frac{n}{2k - 1} \right\rfloor
	\leq n + \sum_{k = 1}^{\left\lceil n/2 \right\rceil} \left\lfloor \frac{n}{2k} \right\rfloor
\]
\[
	\left\lfloor \frac{n}{1} \right\rfloor + \left\lfloor \frac{n}{3} \right\rfloor + ...
	\leq n + \left\lfloor \frac{n}{2} \right\rfloor + \left\lfloor \frac{n}{4} \right\rfloor
	+ \left\lfloor \frac{n}{6} \right\rfloor + ...
\]

Because $n \leq n$, $n/3 \leq n/2$, $n/5 \leq n/4$ etc., this inequality has to hold
in this case.

When the value is negative, then similarly:

\[
	\sum_{k = 1}^{\left\lceil n/2 \right\rceil} \left\lfloor \frac{n}{2k} \right\rfloor
	\leq n + \sum_{k = 1}^{\left\lceil n/2 \right\rceil} \left\lfloor \frac{n}{2k - 1} \right\rfloor
\]
\[
	\left\lfloor \frac{n}{2} \right\rfloor + \left\lfloor \frac{n}{4} \right\rfloor
	+ \left\lfloor \frac{n}{6} \right\rfloor + ...
	\leq n +
	\left\lfloor \frac{n}{1} \right\rfloor + \left\lfloor \frac{n}{3} \right\rfloor
	+ \left\lfloor \frac{n}{5} \right\rfloor+ ...
\]

Because $n/2 \leq n$, $n/4 \leq n$, $n/6 \leq n/3$ etc., the inequality holds in
both cases, so the proof is complete.



\end{document}
