\documentclass{fkssolpub}

\usepackage[czech]{babel}
\usepackage{fontspec}
\usepackage{fkssugar}
\usepackage{amsmath}

\author{Ondřej Sedláček}
\school{Gymnázium Oty Pavla} 
\series{4p}
\problem{1} 

\begin{document}

Firstly, we need to prove that if each animal on the farm gets same
amount of food, the animals cannot be equal. Suppose that animals are
equal when they reach similar level of satisfaction after eating given
amount of food. However, pigs have marginaly bigger minimum food comsumption
than chicken. Therefore, the animals can't be equal when everyone gets the
same amount of food.

Let $c_c$, $c_p$ be amount of food chickens and pigs get. If, on average,
an animal in government gets the same amount of food as an animal outside
of government, the equation:

\[
	\frac{o_c c_c + o_p c_p}{o_c + o_p} = \frac{2 g_c c_c + 2 g_p c_p}{g_c + g_p}
\]

where $o_c$, $o_p$ is the number of chickens and pigs outside of government and
$g_c$, $g_p$ is the number of chickens and pigs in government, has to hold.

So the only thing remaining is to show a case when this equation holds. For
$c_c = 1$, $c_p = 10$, $o_c : o_p = 1 : k$ and $g_c : g_p = k : 1$ we can
show that there exist a real number $k$ which satisfies the relation:

\[
	\frac{o_c + 10 k o_c}{o_c + k o_c} = 2 \cdot \frac{k g_p + 10 g_p}{k g_p + g_p}
\]
\[
	\frac{1 + 10 k}{1 + k} = 2 \cdot \frac{k + 10}{k + 1}
\]
\[
	10k + 1 = 2k + 20
\]
\[
	8k = 19
\]
\[
	k = \frac{19}{8}
\]

So in this case, on average, an animal in government gets the same amount of food as
an animal outside of government, when $o_c : o_p = 8 : 19$ and $g_c : g_p = 19 : 8$.

As this is the only thing we have to show, the proof is complete. Q. E. D.

\end{document}
