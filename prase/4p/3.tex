\documentclass{fkssolpub}

\usepackage[czech]{babel}
\usepackage{fontspec}
\usepackage{fkssugar}
\usepackage{amsmath}

\author{Ondřej Sedláček}
\school{Gymnázium Oty Pavla} 
\series{4p}
\problem{3} 

\begin{document}

Suppose we have an triangle that is inside the square. Using a few steps,
we are able to maximize its area by moving its vertices in certain ways.

In a first step, we select any vertex of the triangle and move it so it's
as far from the opposite edge of the triangle as possible. This way we
will make the area of the triangle bigger. Because the farthest point from
an edge of the triangle has to be in one of the corners of the square, we
will move the vertex to the corner of the square furthest from the edge.

For the second and third vertex of the triangle, we will do the same
steps (except we will not put any of the vertices of the triangle to
already used corner of the square). Then, we are not able to make the area
of the triangle bigger. Hence, it has to be in its maximum. As a result,
the maximum area of the triangle has to be a half of the square.

Q. E. D.

\end{document}
