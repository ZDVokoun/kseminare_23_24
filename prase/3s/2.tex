\documentclass{fkssolpub}

\usepackage[czech]{babel}
\usepackage{fontspec}
\usepackage{fkssugar}
\usepackage{amsmath}
\usepackage{graphicx}

\author{Ondřej Sedláček}
\school{Gymnázium Oty Pavla} 
\series{3s}
\problem{2} 

\begin{document} 

Z cvičení 72 víme, že rovnice tečen kružnice opsané procházející body
$A$, $B$, $C$ jsou postupně:

\[
  b^2 z + c^2 y = 0
\]
\[
  a^2 z + c^2 x = 0
\]
\[
  a^2 y + b^2 x = 0
\]

Při vyřešení soustav rovnic tečen v bodech $A$, $B$ a tečen v bodech
$A$, $C$ dostaneme, že $L_b = (a^2 : -b^2 : c^2)$ a $L_c = (a^2 : b^2 : -c^2)$.
Z úlohy 49 pak víme, že $I_b = (a : -b : c)$ a $I_c = (a : b : -c)$. Obecné
rovnice přímek $L_bI_b$ a $L_cI_c$ jsou postupně:

\[
  (b - c) x + \frac{a (c - a)}{b} y - \frac{a (a - b)}{c} z = 0
\]
\[
  (b - c) x + \frac{a (a - c)}{b} y + \frac{a (a - b)}{c} z = 0
\]

Podle věty 57 si pak pomocí determinantu matice níže ověříme, zda přímky
$L_bI_b$, $L_cI_c$ a $BC$ procházejí jedním bodem:

\[
\begin{vmatrix}
  (b - c) & \frac{a (c - a)}{b} & -\frac{a (a - b)}{c} \\
  (b - c) & \frac{a (a - c)}{b} & \frac{a (a - b)}{c} \\
  1 & 0 & 0
\end{vmatrix} = 
\begin{vmatrix}
  (b - c) & \frac{a (c - a)}{b} & - \frac{a (a - b)}{c} \\
  2(b - c) & 0 & 0 \\
  1 & 0 & 0
\end{vmatrix} = 
\begin{vmatrix}
  0 & \frac{a (c - a)}{b} & - \frac{a (a - b)}{c} \\
  1 & 0 & 0 \\
  0 & 0 & 0
\end{vmatrix} = 0
\]

Tedy přímky procházejí jedním bodem, což jsme chtěli ukázat.

\end{document}
