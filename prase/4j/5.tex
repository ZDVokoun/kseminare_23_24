\documentclass{fkssolpub}

\usepackage[czech]{babel}
\usepackage{fontspec}
\usepackage{fkssugar}
\usepackage{amsmath}

\author{Ondřej Sedláček}
\school{Gymnázium Oty Pavla} 
\series{4j}
\problem{5} 

\begin{document}

\section{Část a}

Protože všechna přirozená čísla jsou rozdělena do aritmetických posloupností,
pak je v nějaké z těchto posloupností zahrnut součin diferencí $d_1 d_2 \cdots
	d_n$. Nechť je tedy zahrnut v posloupnosti $a_n$ a $a_k = d_1 d_2 \cdots d_n$.
Pokud diference této posloupnosti je $d_l$, pak
$a_k \equiv a_k + i d_l \equiv 0 \mod{d_l}$, tedy veškeré čísla této posloupnosti
jsou dělitelná její diferencí, tudíž i její první člen je dělitelný její diferencí.
Tím je důkaz u konce.

\section{Část b}

Jako první si ukážeme, že pokud má nějaký pár z posloupností nesoudělné diference,
pak nemůže být splněna druhá podmínka. Na to nám stačí formulovat soustavu kongruencí,
která musí pro číslo $x$, které se bude nacházet v obou posloupnostech, platit:

\[
	x \equiv a_1 \mod{d_a}
\]
\[
	x \equiv b_1 \mod{d_b}
\]

Čínská zbytková věta nám pak říká, že tato soustava kongruencí má nekonečně mnoho
řešení, pokud jsou čísla $d_a$ a $d_b$ jsou nesoudělná. Proto pokud chceme, aby
tato soustava neměla řešení, musí být diference $d_a$ a $d_b$ nutně soudělná.

Z první podmínky následně nutně musí platit, že od určitého čísla musí být v posloupnostech
zahrnuta všechna čísla. Tím pádem v nějaké posloupnosti se musí nacházet nějaký
násobek součinu diferencí všech posloupností $k \cdot d_1 d_2 \cdots d_{2023}$.
Z části \textit{a} již víme, že všechny členy té posloupnosti, která obsahuje toto číslo,
jsou dělitelná její diferencí. Proto aby platila třetí podmínka, musí být jak první
číslo, tak diference, určité prvočíslo větší než 2023. Aby však platila druhá
podmínka, musí být všechny diference soudělné a tedy být násobkem tohoto prvočísla.

Teď už můžeme sporem ukázat, že pak nemůže platit první podmínka. Tehdy totiž
nejmenší diference bude rovna tomuto prvočíslu větší než 2023, ale protože máme
jen 2023 posloupností, nemůžeme nikdy do těchto posloupností zahrnout
všechna čísla mezi dvěma členy posloupnosti s nejmenší diferencí, protože jejich
počet bude větší než 2022. To tedy rozporuje s předpokladem, že platí první
podmínka, tedy tyto posloupnosti nemůžou existovat.

\end{document}
