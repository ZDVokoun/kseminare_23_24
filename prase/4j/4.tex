\documentclass{fkssolpub}

\usepackage[czech]{babel}
\usepackage{fontspec}
\usepackage{fkssugar}
\usepackage{amsmath}

\author{Ondřej Sedláček}
\school{Gymnázium Oty Pavla} 
\series{4j}
\problem{4} 

\begin{document}

\section{Část a}

Jako první provedeme substituci $x = -y$. Tím pak dostaneme nerovnost:

\[
	y^{2n} + y^{2n - 1} + \dots + y + 1 > \frac{1}{2}
\]

Tato nerovnice zřejmě platí pro všechna $y \geq 0$. Díky tomu můžeme provést
úpravu:

\[
	\frac{y^{2n + 1} - 1}{y - 1} > \frac{1}{2}
\]
\[
	\frac{2y^{2n + 1} - 2 - y + 1}{y - 1} > 0
\]
\[
	\frac{2y^{2n + 1} - y - 1}{y - 1} > 0
\]

Pokud předpokládáme, že $y < 0$, pak platí:

\[
	2y^{2n + 1} - y - 1 < 0
\]
\[
	y (2y^{2n} - 1) < 1
\]

Pro $y \leq - \sqrt[2n]{\frac{1}{2}}$ je levá strana nerovnosti je záporná, tím
pádem nerovnost zřejmě platí. A protože $\sqrt[2n]{\frac{1}{2}} < 1$, musí tato
nerovnost též platit v intervalu $\left\langle - \sqrt[2n]{\frac{1}{2}}; 0 \right)$,
jelikož oba činitele na levé straně budou v tomto intervalu menší než 1.
Proto tedy tato nerovnost platí pro všechna reálná čísla. Tím jsme tuto nerovnost
dokázali.

\end{document}
