\documentclass{fkssolpub}

\usepackage[czech]{babel}
\usepackage{fontspec}
\usepackage{fkssugar}
\usepackage{amsmath}

\author{Ondřej Sedláček}
\school{Gymnázium Oty Pavla} 
\series{4j}
\problem{1} 

\begin{document}

\section{Část a}

Abychom vždy tipli více než polovinu správně stačí nám následující
strategie. Víme, že máme úplný graf hráčů a víme, jaký stupně mají
hráči v podgrafu výher. V každém kroku si vybereme jakéhokoli
hráče $v$ s počtem výher $w_v$. Pak budeme jednat následovně:

\begin{itemize}
	\item Pokud $w_v < \frac{n - 1}{2}$,
	      pak bude pro všechny hry hráče $v$ tipovat prohru.
	\item Pokud $w_v > \frac{n - 1}{2}$,
	      pak bude pro všechny hry hráče $v$ tipovat výhru.
	\item Pokud $w_v = \frac{n - 1}{2}$, pak buď bude tipovat pro
	      všechny hry hráče $v$ výhru, nebo prohru. Nemůže tipovat oboje
	      zároveň.
	\item Jakmile se zeptáme na všechny hry hráče $v$, odstraníme
	      hráče $v$ a jeho hrany z úplného grafu a aktualizujeme počet
	      výher ostatních hráčů podle Samových odpovědí. Tím získáme
	      úplný graf o počtu hráčů $n - 1$ a strategii opakujeme, dokud
	      $n = 1$.
\end{itemize}

Je zřejmé, že pokud máme počet hráčů $n$ sudý, nemůže nastat podmínka
$w_v = \frac{n - 1}{2}$, tím pádem v kroku, kdy je počet hráčů sudý, musíme
tipnout více než polovinu správně. Pro liché $n$ tipneme alespoň polovinu
správně. To však nevadí, protože pro každý graf o počtu grafů $n \geq 2$
získáme alespoň v jednom kroku sudý počet hráčů, tím pádem nutně počet správně
tipnutých odpovědí je více než polovina. Tím je důkaz u konce.


\end{document}
