\documentclass{fkssolpub}

\usepackage[czech]{babel}
\usepackage{fontspec}
\usepackage{fkssugar}
\usepackage{amsmath}
\usepackage{graphicx}

\author{Ondřej Sedláček}
\school{Gymnázium Oty Pavla} 
\series{3j}
\problem{4} 

\begin{document} 

Nejprve uvažujme, že máme balíček $n$ karet splňující podmínky ze
zadání a že chceme tento balíček upravit tak, abychom dostali $n + 1$
karet splňující zadání. Jako první musíme splnit podmínku, že na každý
kartě musí být $n + 1$ symbolů. Protože už v tomto balíčku platí, že každé
dvě karty sdílí právě jeden symbol, nezbývá nám, než na každou kartu
přidat další symbol, čímž vytvoříme celkem $n$ nových symbolů. Jakmile
pak chceme přidat novou kartu, na tuto novou kartu přidáme těch $n$ nově
vytvořených symbolů, což můžeme, protože tyto nové symboly jsou pro každou kartu
původního balíčku jedinečné. Následně vytvoříme nový symbol, abychom 
doplnili počet symbolů na této kartě do $n + 1$. Po tomto kroku
jsme splnili všechny podmínky.

Tím pádem abychom rozšířili
balíček $n$ karet na $n + 1$ karet, musíme přidat alespoň $n + 1$ symbolů.
Tento vztah můžeme vyjádřit rekurentně:

\[
  s(n + 1) = s(n) + n + 1
\]

Zbývá nám tedy se ujistit, že $s(1) = 1$, což zřejmě platí, a zjistíme,
že explicitní vyjádření tohoto vztahu je známý vztah:

\[
  s(n) = \frac{1}{2} n (n + 1)
\]

Tento vztah tedy určuje optimální počet symbolů na balíčku $n$ karet.

\end{document}
