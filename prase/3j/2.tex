\documentclass{fkssolpub}

\usepackage[czech]{babel}
\usepackage{fontspec}
\usepackage{fkssugar}
\usepackage{amsmath}
\usepackage{graphicx}

\author{Ondřej Sedláček}
\school{Gymnázium Oty Pavla} 
\series{3j}
\problem{2} 

\begin{document} 

Víme, že Majda vzala celkem 302 čísel. Díky tomu, že $0 + 1 = 1$,
můžeme snadno snížit počet uvažovaných čísel na 300, a to na čísla
2, 3, ..., 301. A protože těchto čísel je sudý počet a jedná se o
lineární posloupnost, snadno můžeme tyto čísla spárovat tak, aby všechny
součty byli rovny 303. Když je pak vynásobíme mezi sebou, získáme:

\[
  1 \cdot 303^{150} = (303^{10})^{15}
\]

Tudíž se jedná o patnáctou mocninu celého čísla, což jsme chtěli ukázat.

\end{document}
