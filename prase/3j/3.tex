\documentclass{fkssolpub}

\usepackage[czech]{babel}
\usepackage{fontspec}
\usepackage{fkssugar}
\usepackage{amsmath}

\author{Ondřej Sedláček}
\school{Gymnázium Oty Pavla} 
\series{3j}
\problem{3} 

\begin{document}

Áďa může označit políčka tak, aby Ben nezvládl projít mezi
stejně označenými políčky. Pro $n = 16$ může Áďa vyplnit tabulku
takto:

\begin{table}[h!]
	\label{tab:1}
	\centering
	\begin{tabular}[c]{|c|c|c|c|c|c|}
		\hline
		   & 13 & 14 & 15 & 16 &   \\
		\hline
		12 & 1  & 2  & 3  & 4  & 1 \\
		\hline
		11 & 5  & 6  & 7  & 8  & 2 \\
		\hline
		10 & 9  & 10 & 11 & 12 & 3 \\
		\hline
		9  & 13 & 14 & 15 & 16 & 4 \\
		\hline
		   & 8  & 7  & 6  & 5  &   \\
		\hline
	\end{tabular}
	\caption{Řešení pro $n = 16$}
\end{table}


\end{document}
