\documentclass{fkssolpub}

\usepackage[czech]{babel}
\usepackage{fontspec}
\usepackage{fkssugar}
\usepackage{amsmath}
\usepackage{graphicx}

\author{Ondřej Sedláček}
\school{Gymnázium Oty Pavla} 
\series{3j}
\problem{5} 

\begin{document} 

Nechť jsou $M_a$ a $M_b$ množina matfyzaček, se kterými tancoval
matfyzák $a$ a se kterými tancoval matfyzák $b$. Aby platilo,
že pro tuto dvojici neexistují matfyzačky $\alpha$ a $\beta$, pro které
platí podmínka v zadání, musí platit, že:

\[
  (M_a \setminus M_b = \emptyset) \lor (M_b \setminus M_a = \emptyset)
\]

Výrok výše platí, když jedna z množin $M_a$ a $M_b$ je podmnožina té druhé.
Proto pokud pro žádnou dvojici matfyzáků neexistuje dvojice matfyzaček,
kterou hledáme, umíme seřadit matfyzáky tak, že bude pro ně platit:

\[
  M_1 \subseteq M_2 \subseteq ... \subseteq M_n
\]

A poněvadž víme, že každá matfyzačka tancovala s alespoň jedním matfyzákem,
musel by $n$-tý matfyzák tancovat se všemi matfyzačkami. To je ale spor.

Tím je důkaz u konce.

\end{document}
