\documentclass{fkssolpub}

\usepackage[czech]{babel}
\usepackage{fontspec}
\usepackage{fkssugar}
\usepackage{amsmath}

\author{Ondřej Sedláček}
\school{Gymnázium Oty Pavla} 
\series{1p}
\problem{6} 

\begin{document}

Pro tuto úlohu budu uvažovat graf $G$, jehož vrcholy budou tvořit mrakodrapy
a hrany budou různé rozdíly výšek o velikosti mocniny dvojky, tudíž pokud jsou
dva páry mrakodrapů se stejným rozdílem výšek, bude jen jeden z párů spojen
hranou. Když dokážeme, že graf $G$ je strom, pak počet různých mocnin dvojky
může být nejvýše $n - 1$. To se pokusím dokázat sporem.

Předpokládejme, že graf $G$ má v sobě cyklus, tudíž se nejedná o strom. V
tomto cyklu vybereme mrakodrapy s největší výškou a nejmenší výškou. Pokud
se jedná o cyklus, pak to znamená, že musí existovat pro rozdíl mezi
největší a nejmenší výškou dva zápisy pomocí součtů a rozdílů mocnin dvojky,
kde však každá mocnina dvojky bude nejvýše jednou a kde žádná mocnina dvojky
nebude větší než rozdíl samotný. To však není možné, protože čísla lze ve
dvojkový soustavě zapsat jen jedním způsobem a při použití rozdílu mocnin
dvojky jsme nuceni použít mocninu větší než samotný rozdíl. A to je spor,
proto graf $G$ musí být nutně strom.

Největší počet rozdílných mocnin dvojky je proto $n - 1$ (příklad můžou být
mrakodrapy o výšce $2^0, 2^1, ..., 2^n$).


\end{document}
