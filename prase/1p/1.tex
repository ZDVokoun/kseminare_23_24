\documentclass{fkssolpub}

\usepackage[czech]{babel}
\usepackage{fontspec}
\usepackage{fkssugar}

\author{Ondřej Sedláček}
\school{Gymnázium Oty Pavla} 
\series{1p}
\problem{1} 

\begin{document}

Protože trojúhelníky $ADE$ a $BDE$ sdílí stejnou výšku $AB;E$, musí platit
$|AD| = |BD|$. Z této rovnosti vyplývá, že musí $2|AE;D| = |CE;B|$. Proto
taky platí $2|CE| = |AE|$, z čehož už jsme schopni získat požadovaný poměr:

\[
	\frac{|AE|}{|AC|} = \frac{|AE|}{|AE| + |CE|} = \frac{2|CE|}{3|CE|} = \frac{2}{3}
\]

\end{document}
