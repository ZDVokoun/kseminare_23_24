\documentclass{fkssolpub}

\usepackage[czech]{babel}
\usepackage{fontspec}
\usepackage{fkssugar}

\author{Ondřej Sedláček}
\school{Gymnázium Oty Pavla} 
\series{1p}
\problem{5} 

\begin{document}

Zde použiji důkaz sporem. Předpokládejme, že existuje graf $G$, u kterého
Pepa nemá šanci zdolat celou trasu a zároveň splňuje zadání.

Pro potřeby důkazu použiju algoritmus, u kterého si vyberu směr cesty a pak
zjednodušuji zadaný graf tak, že v každém kroku si najdu hranu
$e_{i,i+1}$, kterou je Pepa schopen ze
zásob vody ve vrcholu $v_i$ překonat, a pak ji spolu s vrcholy $v_i$ a $v_{i+1}$
nahradíme novým vrcholem se zásoby vody o velikosti $|v_i| + |v_{i+1}| - |e_{i,i+1}|$,
dokud budou hrany k nahrazování. Tento algoritmus zachovává vlastnost, že ve studních
je přesně tolik vody, kolik je potřeba ke zdolání celé cesty.

Když do tohoto algoritmu dáme jako vstup
graf $G$, dostaneme graf $G'$, kde ale pro každý vrchol $v_i$ a hranu
$e_{i,i+1}$ platí $v_i < e_{i, i+1}$, což je spor. Tím je důkaz u konce.


\end{document}
