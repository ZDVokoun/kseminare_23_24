\documentclass{fkssolpub}

\usepackage[czech]{babel}
\usepackage{fontspec}
\usepackage{fkssugar}

\author{Ondřej Sedláček}
\school{Gymnázium Oty Pavla} 
\series{1p}
\problem{4} 

\begin{document}

Ze zadání nutně vyplývá, že polynom $P(x)$ musí mít nutně dva různé kořeny,
jinak polynom $P(Q(x))$ nemůže mít čtyři rúzné kořeny. Dále kořeny polynomu
$P(x)$ musí být čísla $Q(x_1)$, $Q(x_2)$, $Q(x_3)$ a $Q(x_4)$, ale protože
má $P(x)$ jen dva kořeny, musí si být některé z těchto čísel rovny. Zde využijeme
toho, že graf kvadratické funkce je osově souměrný a platí pro ni
$Q(c - x) = Q(c + x)$ pro určité reálné číslo $c$. Pak tedy kvůli nerovnosti
ze zadání musí platit $Q(x_1) = Q(x_4)$ a $Q(x_2) = Q(x_3)$, díky čemuž můžeme
použít substituci $x_1 = c - x_{14}$, $x_2 = c - x_{23}$, $x_3 = c + x_{23}$ a
$x_4 = c + x_{14}$. Po dosazení a zjednodušení dostaneme:

\[
	x_1 + x_4 = x_2 + x_3
\]
\[
	c - x_{14} + c + x_{14} = c - x_{23} + c + x_{23}
\]
\[
	2c = 2c
\]

Tím je důkaz u konce.

\end{document}
