\documentclass{fkssolpub}

\usepackage[czech]{babel}
\usepackage{fontspec}
\usepackage{fkssugar}
\usepackage{amsmath}
\usepackage{graphicx}

\author{Ondřej Sedláček}
\school{Gymnázium Oty Pavla} 
\series{2s}
\problem{2} 

\begin{document} 

Nechť je počátek komplexní roviny ve středu kružnice opsané sedmiúhelníku,
která je jednotková, a bod $a = 1$. Jelikož pro orthocentrum trojúhelníku 
$BCE$ platí $h = b + c + e$ a vrcholy sedmiúhelníku můžeme vyjádřit jako
mocniny sedmé odmocniny z jedné $\xi$ (pro ujasnění, $\xi = \xi_1
 = \cos\left(\frac{2\pi}{7}\right) + i \sin\left(\frac{2\pi}{7}\right)$ podle
značení v seriálu), je tedy $h = \xi + \xi^2 + \xi^4$. Jednu z možných
hodnot $h$ jsme pak schopni zjistit tak, že zjistíme, jakou kvadratickou
rovnici $h$ splňuje:

\begin{gather*}
  h^2 = (\xi + \xi^2 + \xi^4)^2 = \xi^2 + \xi^4 + \xi^8 + 2(\xi \xi^2
   + \xi \xi^4 + \xi^2 \xi^4) = \xi + \xi^2 + \xi^4 + 2(\xi^3 + \xi^5
   + \xi^6) \\ = h - 2(1 + \xi + \xi^2 + \xi^4) = h - 2(1 + h) = - h - 2
\end{gather*}
\[
  h^2 + h + 2 = 0
\]

Tuhle kvadratickou rovnici nám zbývá vyřešit v komplexních číslech:

\[
  D = 1^2 - 4 \cdot 2 = -7
\]
\[
  h = \frac{-1 \pm \sqrt{-7}}{2} = -\frac{1}{2} \pm i \frac{\sqrt{7}}{2}
\]

Máme tedy dvě různá řešení, ale protože jsou obě stejně vzdálená od
bodu $A$, nemusí nás to zajímat a rovnou vypočítáme vzdálenost $|AH|$:

\[
  |AH| = |h - a| = \sqrt{\frac{9}{4} + \frac{7}{4}} = \sqrt{\frac{16}{4}} = 2
    = 2r
\]

Protože jsme kružnici opsanou zvolili na začátku jako jednotkovou, dokázali
jsme tvrzení ze zadání.

\end{document}
