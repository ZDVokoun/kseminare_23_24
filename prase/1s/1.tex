\documentclass{fkssolpub}

\usepackage[czech]{babel}
\usepackage{fontspec}
\usepackage{fkssugar}
\usepackage{amsmath}
\usepackage{graphicx}

\author{Ondřej Sedláček}
\school{Gymnázium Oty Pavla} 
\series{1s}
\problem{1} 

\begin{document} 

Nechť $\alpha$ je úhel u vrcholu A a $\beta$ úhel u vrcholu B. Obsah tětivového
čtyřúhelníku ABCD můžeme získat dvěma způsoby:

\[
  S_{ABCD} = \frac{1}{2} \left(|AB| \cdot |DA| \cdot \sin \alpha + |BC| \cdot |CD| \cdot \sin (180^{\circ} - \alpha) \right)
    = \frac{1}{2} \left(|AB| \cdot |BC| \cdot \sin \beta + |DA| \cdot |CD| \cdot \sin (180^{\circ} - \beta) \right)
\]

Tuto rovnost můžeme dále upravit:

\[
  \sin \alpha \cdot (|AB| \cdot |DA| + |BC| \cdot |CD|)
   = \sin \beta \cdot (|AB| \cdot |BC| + |DA| \cdot |CD|)
\]
\[
  \frac{\sin \beta}{\sin \alpha} = \frac{|AB| \cdot |DA| + |BC| \cdot |CD|}
    {|AB| \cdot |BC| + |DA| \cdot |CD|}
\]

Poněvadž trojúhelníky $ABC$ a $ABD$ mají stejné kružnice opsané, ze sinové věty
máme rovnost:

\[
  \frac{|BD|}{\sin \alpha} = \frac{|AC|}{\sin \beta}
\]

Po dosazení konečně dostaneme:

\[
  \frac{|AC|}{|BD|} = \frac{|AB| \cdot |DA| + |BC| \cdot |CD|}
    {|AB| \cdot |BC| + |DA| \cdot |CD|}
\]

\end{document}
