\documentclass{fkssolpub}

\usepackage[czech]{babel}
\usepackage{fontspec}
\usepackage{fkssugar}
\usepackage{amsmath}
\usepackage{graphicx}

\author{Ondřej Sedláček}
\school{Gymnázium Oty Pavla} 
\series{1p}
\problem{2} 

\begin{document} 

Nechť je $D$ bod dotyku tečny a kružnice $\omega_2$. Z tvrzení 86 
nutně vyplývá, že $|BD|$ a $|DC|$ jsou vlny. Z důkazu tohoto tvrzení
pak vyplývá to, že i $|AB|$ a $|AC|$ jsou vlny. Proto když najdeme
bod, kdy je $|BC|$ i $|AB| + |BC| + |AC|$ nulový, máme dokázáno.
A k tomu dojde právě když $D = A$. Tehdy jsou oba výrazy nulový,
tím pádem mají konstatní poměr. Q. E. D.

\end{document}
