\documentclass{fkssolpub}

\usepackage[czech]{babel}
\usepackage{fontspec}
\usepackage{fkssugar}
\usepackage{amsmath}

\author{Ondřej Sedláček}
\school{Gymnázium Oty Pavla} 
\series{2p}
\problem{2} 

\begin{document}

Jako první upravíme výraz ze zadání do tohoto tvaru:

\[
	2023 \cdot 2025 \mid 2023a + 2025b
\]

Tedy musí nutně platit:

\[
	2023 \mid 2023a + 2025b
\]
\[
	2025 \mid 2023a + 2025b
\]

A protože čísla 2023 a 2025 jsou nesoudělná ($2023 = 7 \cdot 17^2$ a
$2025 = 3^4 \cdot 5^2$), musí platit $a = 2025 a'$, $b = 2023 b'$, kde
$a'$ a $b'$ jsou celá čísla. Pak tedy:

\[
	2024^2 - 1 \mid 2025 a' \cdot 2023 b'
\]
\[
	2023 \cdot 2025 \mid 2023 \cdot 2025 \cdot a'b'
\]

Což zřejmě platí. Tím je důkaz u konce.

\end{document}
