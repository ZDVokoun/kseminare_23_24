\documentclass{fkssolpub}

\usepackage[czech]{babel}
\usepackage{fontspec}
\usepackage{fkssugar}
\usepackage{amsmath}

\author{Ondřej Sedláček}
\school{Gymnázium Oty Pavla} 
\series{2p}
\problem{3} 

\begin{document} 

Nechť je číslo $n$ nejmenší číslo posloupnosti po sobě jdoucích 39 čísel, které
je dělitelné deseti a zároveň na místě desítek nemá číslovku 9. Toto číslo může
v nejhorším případě mít index 20, proto je vždycky prvkem této posloupnosti taky 
číslo $n+10$. Protože
nemůže dojít k "přetečení" kvůli devítce na místě desítek, ciferné součty čísel
$n$ a $n + 10$ se liší o jedna. Proto nemůžou být kongruentní
s jednou modulo 11 ciferné součty obou čísel zároveň, tím pádem stačí správně nastavit 
jednomu z těchto čísel cifru na jednotkovém místě
a nalezli jsme číslo, jehož ciferný součet je dělitelný jedenácti. A poněvadž
k číslu $n$ přičteme tímto postupem v nejhorším případě 19, je nalezené číslo
prvkem naší posloupnosti.

Q. E. D.

\end{document}
