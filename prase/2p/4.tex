\documentclass{fkssolpub}

\usepackage[czech]{babel}
\usepackage{fontspec}
\usepackage{fkssugar}
\usepackage{amsmath}

\author{Ondřej Sedláček}
\school{Gymnázium Oty Pavla} 
\series{2p}
\problem{4} 

\begin{document}

Pro získání součtu všech čísel na tabuli uvažujme následující algoritmus.
Jako první spočítáme součet aritmetické posloupnosti $n + 1, n + 2, ...,
	2n$. Následně pro každé sudé číslo provedeme to, že vytvoříme novou posloupnost,
která zahrnuje poloviny sudých čísel předchozí posloupnosti, součet nové
posloupnosti následně odečteme a tento postup budeme opakovat na nové
posloupnosti, dokud máme v posloupnosti sudá čísla.

Důležité je však následující pozorování -- pokud bychom sloučili všechny ty
posloupnosti vytvořené tím algoritmem podle velikosti, dostaneme posloupnost
$1, 2, ..., n$. Prvky v této posloupnosti se nemohou opakovat, protože
pro celé $n$ je $2n$ vždy jedinečné, a zároveň pro posloupnost $n + 1,
	n + 2, ..., 2n$ a pro posloupnost $n + 1, n + 2, ..., 2n + 1$ v jednom
kroku dostaneme posloupnost $\left\lfloor \frac{n}{2} \right\rfloor + 1, ..., n$.

Tím pádem součet čísel na tabuli bude:

\[
	n \cdot \frac{3n + 1}{2} - n \cdot \frac{n + 1}{2}
	= \frac{n}{2} (3n + 1 - n - 1) = n^2
\]

\end{document}
