\documentclass{fkssolpub}

\usepackage[czech]{babel}
\usepackage{fontspec}
\usepackage{fkssugar}
\usepackage{amsmath}
\usepackage{graphicx}

\author{Ondřej Sedláček}
\school{Gymnázium Oty Pavla} 
\series{2}
\problem{1j} 

\begin{document} 

Jako první dokážeme, že součástí této posloupnosti jsou mocniny
prvočísel $p^{\alpha} > 1$. Díky podmínce ze zadání je zřejmé, 
že posloupnost ${a_n}_{n=1}^{\infty}$ je rostoucí. Proto mocnina
prvočísla $p^{\alpha}$, kterou chceme přidat, je nutně větší než všechna
čísla v posloupnosti. A protože každé číslo menší než $p^{\alpha}$ má
v prvočíselném rozkladu prvočíslo $p$ v menší mocnině než $\alpha$, 
splňuje $p^{\alpha}$ podmínku posloupnsti a je nutně její součástí.



\end{document}
