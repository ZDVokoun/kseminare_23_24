\documentclass{fkssolpub}

\usepackage[czech]{babel}
\usepackage{fontspec}
\usepackage{fkssugar}
\usepackage{amsmath}
\usepackage{graphicx}

\author{Ondřej Sedláček}
\school{Gymnázium Oty Pavla} 
\series{2}
\problem{1j} 

\begin{document}

Jako první dokážeme, že součástí této posloupnosti jsou mocniny
prvočísel $p^{\alpha} > 1$ (jednička už je součástí posloupnosti,
proto ta podmínka). Díky podmínce ze zadání je zřejmé,
že posloupnost $\{a_n\}_{n=1}^{\infty}$ je rostoucí, protože při
přidání menšího čísla, které nebylo přidáno do posloupnosti,
se nutně nemůže zvětšit nejmenší společný násobek. Proto mocnina
prvočísla $p^{\alpha}$, kterou chceme přidat, je nutně větší než všechna
čísla v posloupnosti. A protože každé číslo menší než $p^{\alpha}$ má
v prvočíselném rozkladu prvočíslo $p$ v menší mocnině než $\alpha$,
splňuje $p^{\alpha}$ podmínku posloupnsti a je nutně její součástí.

Teď ukážeme, že žádná jiná čísla nemohou být součástí posloupnosti.
Předpokládejme, že můžeme do posloupnosti přidat číslo $x = p_1^{\alpha_1}
	\cdot p_2^{\alpha_2} \cdots p_k^{\alpha_k}$. Protože každé z čísel
$p_1^{\alpha_1}$, $p_2^{\alpha_2}$, ..., $p_k^{\alpha_k}$ jsou
menší než číslo $x$ a jsou součástí posloupnosti (jednička je
taky součástí posloupnosti), toto číslo nemůže zvětšit nejmenší
společný násobek všech čísel a tedy není součástí posloupnosti.

Tudíž součástí posloupnosti jsou všechny mocniny prvočísel. Q. E. D.

\end{document}
