\documentclass{fkssolpub}

\usepackage[czech]{babel}
\usepackage{fontspec}
\usepackage{fkssugar}
\usepackage{amsmath}
\usepackage{graphicx}

\author{Ondřej Sedláček}
\school{Gymnázium Oty Pavla} 
\series{1j}
\problem{4} 

\begin{document} 

Abychom ukázali, že tato posloupnost je neklesající, stačí nám
ukázat, že umíme jednosměrně spárovat kombinace, jak poskládat váhu
$n$, s kombinacemi, jak poskládat váhu $n + 1$. To uděláme tak,
že ukážeme způsob, jak každou kombinaci vah tvořící váhu $n$
upravit tak, abychom dostali jedinečnou kombinaci vah tvořící váhu
$n + 1$.

To zvládneme vcelku jednoduše. Budeme to řešit pro dva různé typy
kombinací -- ty, který obsahují váhu 1, a ty, který ji naopak neobsahují.
Pro kombinaci vah, která neobsahují váhu 1, ji získáme jednoduše
jejím přidáním. Pro kombinaci vah obsahující váhu 1 novou kombinaci
získáme tak, že váhu 1 odstraníme a místo nejtěžší váhy $m$ v původní
kombinaci dáme váhu $m + 2$. To můžeme udělat, protože pokud kombinace
obsahuje váhu 1, pak zároveň obsahuje i další váhy, protože posloupnost
neobsahuje prvek $a_1$.

Oba způsoby úprav zachovávají všechny potřebné vlastnosti -- všechny
váhy jsou liché, každá je zastoupena nejvýše jednou a každá kombinace
je jedinečná. Tím jsme získali platné párování a důkaz je tedy u konce.


\end{document}
