\documentclass{fkssolpub}

\usepackage[czech]{babel}
\usepackage{fontspec}
\usepackage{fkssugar}
\usepackage{amsmath}

\author{Ondřej Sedláček}
\school{Gymnázium Oty Pavla} 
\series{1j}
\problem{3} 

\begin{document}

Uvažujme jakoukoli posloupnost splňující podmínky ze zadání. Abychom
ukázali, že posloupnost bude od jistého bodu monotónní, stačí
nám ukázat, že existuje jen jedna dvojice po sobě jdoucích prvků,
jejichž monotónní podposloupnosti mají překryv jen jeden prvek. Pokud
totiž existuje jen jedna taková dvojice, pak je zřejmé, že posloupnost
je od jistého bodu monotónní.

Pokud totiž podposloupnost $i$-tého prvku zahrnuje i první prvek,
pak jediný prvek, který podposloupnost zahrnuje a který následuje
po $i$-tém prvku, je $i+1$-tý prvek. Proto tedy, když podposloupnost
$i+1$-tého prvku začíná právě $i+1$-tým prvkem, pak podposloupnost
tohoto prvku zahrnuje alespoň $i+1$ následujících prvků. Protože to
samé platí i o podposloupnostech následujících prvků, je tedy překryv
těchto podposloupností alespoň 2, tím pádem od $i+1$ tého prvku je
celá posloupnost monotónní, což jsme chtěli ukázat. Q. E. D.

\end{document}
