\documentclass{fkssolpub}

\usepackage[czech]{babel}
\usepackage{fontspec}
\usepackage{fkssugar}
\usepackage{amsmath}
\usepackage{graphicx}

\author{Ondřej Sedláček}
\school{Gymnázium Oty Pavla} 
\series{1j}
\problem{1} 

\begin{document} 

Příklad takové posloupnosti je posloupnost:

\[
  1, 4, 1, 1, 4, 1, 1, 4, 1, 1, ..., 4, 1, 1
\]

Zde je jasně vidět, že jakmile vybereme jakýkoli úsek jedniček, musíme
pak přičíst čtyřku. A protože každý úsek jedniček má součet menší než tři,
splňuje podmínku, že žádný úsek nemá součet 3.

Rychle jsme si také schopni zkontrolovat, že má přesně součet, který chceme.
Součet celé posloupnosti je $337 \cdot 4 + 675 \cdot 1 = 2023$, takže splňuje
jak délku, tak i součet posloupnosti.

\end{document}
