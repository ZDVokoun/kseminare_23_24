\documentclass{fkssolpub}

\usepackage[czech]{babel}
\usepackage{fontspec}
\usepackage{fkssugar}
\usepackage{amsmath}
\usepackage{graphicx}

\author{Ondřej Sedláček}
\school{Gymnázium Oty Pavla} 
\series{1j}
\problem{5} 

\begin{document} 

Předpokládejme, že jeden z prvků $v_n = \frac{p}{q}$, kde $p, q$ jsou
přirozená čísla, pro které platí $\gcd(p, q) = 1$.

Pak po dosazení dostaneme:

\[
  v_{n+1} = \frac{p}{q} + \frac{2}{\frac{p}{q}} = \frac{p^2 + 2 q^2}{pq}
\]

Protože číslo 2024 není čtverec, jmenovatel čísla $v_2$ také není čtverec.
Když se nám tedy povede ukázat, že pokud jmenovatel předchozího členu není
čtverec, tak ani následující člen nebude mít jmenovatele čtverec, 
máme důkaz hotový. Pro to však musíme zjistit, jestli zlomek
výše je v základním tvaru:

\begin{gather*}
  \gcd(p^2 + 2 q^2, pq) = \gcd(p^2 - 3pq + 2q^2, pq) = \gcd((p - q)(p - 2q), pq) \\
    = \gcd(p-q, p) \cdot \gcd(p-2q, p) \cdot \gcd(p-q, q) \cdot \gcd(p-2q, q) \\
    = \gcd(p-2q, p) = \gcd(2q, p) = \gcd(p, 2)
\end{gather*}

Tedy můžou být jen dva případy -- buď je tento zlomek už v základním tvaru,
nebo bude vůči zlomku v základním tvaru rozšířen o dvojku. Předpokládejme
tedy, že $q$ není čtverec. Pokud je zlomek v základním tvaru, je zřejmé,
že jmenovatel $pq$ nebude čtverec, protože čísla $p, q$ jsou nesoudělná.
V druhém případě to je podobné -- jmenovatel $pq/2$ nebude čtverec, protože
$2 \nmid q$, $2 \mid p$ a čísla $p, q$ jsou nesoudělná.

Indukční krok tedy platí, čímž je důkaz u konce.

\end{document}
