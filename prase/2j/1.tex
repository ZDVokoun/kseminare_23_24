\documentclass{fkssolpub}

\usepackage[czech]{babel}
\usepackage{fontspec}
\usepackage{fkssugar}
\usepackage{amsmath}

\author{Ondřej Sedláček}
\school{Gymnázium Oty Pavla} 
\series{2j}
\problem{1} 

\begin{document}

Počet účastníku, se kterými se Lenka kamarádí, zjistíme snadno. Ten účastník,
který se kamarádí s 23 účastníky, se kamarádí se všemi. Tudíž se kamarádí
jak s Lenkou, tak i s účastníkem, který má jen jednoho kamaráda. Tedy jakmile
odstraníme vrchol účastníka s 23 kamarády a vrchol s jedním kamarádem, dostaneme
podobný případ, ale teď budeme řadit 21 účastníků.

Tento postup, kdy odstraníme účastníka s nejvíce kamarády a účastníka s jedním
kamarádem, opakujeme do té doby, než nám zbyde jediný účastník, který se
nutně kamarádí s Lenkou. Tedy počet účastníků, se kterými se Lenka kamarádí,
je 12 (výsledek dostaneme jako $\lfloor 23/2 \rfloor + 1$).

\end{document}
