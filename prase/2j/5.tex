\documentclass{fkssolpub}

\usepackage[czech]{babel}
\usepackage{fontspec}
\usepackage{fkssugar}
\usepackage{amsmath}

\author{Ondřej Sedláček}
\school{Gymnázium Oty Pavla} 
\series{2j}
\problem{5} 

\begin{document}

Po celý důkaz bude skupinka přátel označovat souvislou komponentu grafu prasátek,
kde hrany označují, že jsou dané prasátka přátelé. Teď ukážu, že skupinka přátel
musí tvořit úplný podgraf grafu přátelství.

Pro skupinku přátel s jedním a se dvěma prasátky je zřejmé, že skupinka přátel
tvoří úplný podgraf. Pak předpokládajme, že skupinka $n$ přátel tvoří úplný
podgraf. Pak pokud budeme chtít přidat další prasátko do skupinky přátelství,
musí jeden z prasátek v skupince přátel se kamarádit s tímto dalším prasátkem.
Ale pak se s ním musí kamarádit i zbytek skupinky přátel, protože nepřítel
přítele je nepřítel. Proto i skupinka $n+1$ přátel je také úplný podgraf,
což jsme chtěli.

Ještě potřebujeme ukázat, že každá skupinka přátel musí být stejně velká,
abychom splnili podmínku, že každý má 2024 nepřátel. To lze jednoduše ukázat
-- pokud by existovaly dvě skupinky přátel, které mají rozdílný počet prasátek,
pak prasátka v těchto skupinkách mají rozdílný počet nepřátel. Tehdy však
nemůžeme splnit podmínku, že každý má 2024 nepřátel, čímž jsme došli ke sporu.

Tím pádem $n$ může být jen takové, kdy jsou splněny dokázané podmínky výše a
podmínka, že každý má 2024 nepřátel. To splňují všechna čísla $n = 2024 + k$,
kde $k \mid 2024$. Všechny ostatní čísla nemohou tyto podmínky splnit.

\end{document}
