\documentclass{fkssolpub}

\usepackage[czech]{babel}
\usepackage{fontspec}
\usepackage{fkssugar}
\usepackage{amsmath}

\author{Ondřej Sedláček}
\school{Gymnázium Oty Pavla} 
\series{2p}
\problem{4} 

\begin{document}

Protože můžeme typy žetonů označovat jakkoliv, označuju je prvními písmeny,
tím pádem písmeno \verb|č| znamená červenou, \verb|m| znamená modrou atd.

\section{Část a}

Důležité je si uvědomit, že červených žetonů musí být ostře více než modrých
žetonů. Proto výsledný program bude:

\begin{verbatim}
č, m -> 2t
m -> t
č -> r
r, t -> 2r
\end{verbatim}

Když je červených více, vzniklé tyrkysové žetony budou nahrazeny, jinak
zůstanou tyrkysové žetony. To splňuje ostrou nerovnost v zadání. Zároveň
nutně nemůžou červený a modrý zůstat.

\section{Část b}

Tady si pro pomoc zavedu podmínky $z \leq 1$ a $\text{\textit{ž}} \leq 1$. Zelený žeton
zajistí, aby k příkazu inicializace (první příkaz) došlo jen jednou,
následně použiju tmavě červené a tmavě modré žetony jako přechodné žetony
a pak žlutý žeton nastartuje převod tmavě červených a tmavě modrých žetonů
na červené, modré a fialové žetony. Kód programu je:

\begin{verbatim}
č, m -> tč, tm, z
tč, ž -> č, f, ž
tm, ž -> m, f, ž
č, tč, z -> 2tč, z
m, tm, z -> 2tm, z
z -> z, ž
\end{verbatim}

Program se nezacyklí díky podmínkám výše.

\section{Část c}

Tady zelený žeton vznikne z červeného žetonu a slouží k odstartování kola,
během kterého vzniknou tmavě modré žetony ze všech modrých a z nich pak
vzniknou modré a fialové žetony. Během toho kola se zelený žeton odstraní a
zároveň když už zůstanou jen fialové a modré žetony, modré žetony se všechny
smažou, tím pádem zůstanou jen fialové žetony. Takže to funguje jako přičítání
$m$ $c$-krát.

\begin{verbatim}
m, z -> tm, z
tm, z -> m, f
tm -> m, f
č -> z
m -> 
\end{verbatim}

\end{document}
