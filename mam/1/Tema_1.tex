\documentclass{fkssolpub}

\usepackage[czech]{babel}
\usepackage{fontspec}
\usepackage{fkssugar}
\usepackage{amsmath}

\author{Ondřej Sedláček}
\school{Gymnázium Oty Pavla} 
\series{1}
\problem{1} 

\begin{document}

\section{Úloha 2}

U velčiny práce $W$ si musíme uvědomit to, že velčina $\mathbf{s}$ je
vzdálenost výchozího a koncovýho bodu, nikoliv dráha. Ta dráha může být
různá, ale pokud je výchozí a koncový bod stejný, je práce stejná, protože
se stav hmotného bodu změnil stejně. Proto je práce skalární.

\section{Problém 3}

Ano, práce plynu $W$ může být záporná, protože plyn je stačitelný. Tato práce
se někdy označuje jako $W' = - W$, která označuje práci vykonanou na plynu.

\section{Úloha 4}

\subsection{První}

Ze zadání víme, že tento děj je izotermický, proto nutně platí:

\[
	p V = \text{konst.}
\]

Proto nový objem bubliny bude:

\[
	V' = \frac{p V}{p'}
\]

Když dosadíme za objem vzorec pro objem koule, dostaneme:

\[
	\frac{4}{3} \pi r'^3 = \frac{4 p}{3 p'} \pi r^3
\]
\[
	r'^3 = \frac{p}{p'} r^3
\]
\[
	\frac{r'}{r} = \sqrt[3]{\frac{p}{p'}}
\]

Teď zbývá dosadit hodnoty:

\[
	\frac{r'}{r} = \sqrt[3]{\frac{p' + h \rho g}{p'}}
	= \sqrt[3]{\frac{101325 + 22 \cdot 998 \cdot 9{,}81}{101325}}
	\doteq 1{,}46
\]

\subsection{Druhý}

Teď musíme pracovat i s teplotou soustavy, ale stále platí:

\[
	\frac{p V}{T} = \text{konst.}
\]

Z čehož lze odvodit:

\[
	V' = \frac{p V T'}{p' T}
\]

\[
	\frac{r'}{r} = \sqrt[3]{\frac{p T'}{p' T}}
\]

Po dosazení dostaneme:

\[
	\frac{r'}{r} = \sqrt[3]{\frac{(101325 + 22 \cdot 998 \cdot 9{,}81) \cdot (273{,}15 + 37)}{101325 \cdot (273,15 + 35)}}
	\doteq 1{,}465
\]

Jak je vidět, malý rozdíl zde ve výsledku je.

\subsection{Třetí}

Protože se jedná o izotermní děj, pro zjištění práce vykonané bublinou $W'$ nám
stačí zjistit hodnotu tohoto určitého integrálu:

\[
	\int^{V'}_{V} p \, dV = \int^{V'}_{V} \frac{p' V'}{V} \, dV
	= p' V' \ln \frac{V'}{V} = p' V' \ln \frac{p}{p'}
\]
\[
	p' V' \ln \frac{p}{p'}
	= 101325 \cdot \frac{4}{3} \pi \cdot 3{,}5^3 \cdot 10^{-18} \cdot \ln \frac{101325 + 22 \cdot 998 \cdot 9{,}81}{101325}
	\doteq 2{,}07 \cdot 10^{-11} \text{J}
\]

V realitě by tento děj nebyl izotermní, protože v hloubce 22 m bude s chladněji
než při hladině a zároveň ten děj neproběhne okamžitě, proto dojde k přenosu
tepla mezi mořem a bublinou.


\end{document}
