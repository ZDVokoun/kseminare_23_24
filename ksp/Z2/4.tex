\documentclass{fkssolpub}

\usepackage[czech]{babel}
\usepackage{fontspec}
\usepackage{fkssugar}
\usepackage{amsmath}
\usepackage{graphicx}

\author{Ondřej Sedláček}
\school{Gymnázium Oty Pavla} 
\series{Z2}
\problem{4} 

\begin{document} 

Jako první začneme odstraňováním zpomalováků. Označme nejmenší rychlost
na určitém místě na trati $k$ (tolerujeme zápornou rychlost). Pak dolní odhad
počtu zpomalováků k odstranění je $1 - k$. Tohoto počtu odstranění zpomalováků
jsme taky schopni dosáhnout, a to tím, že začneme od začátku a vždy odstraníme
zpomalovák, po kterým by vlak zastavil. Tento postup dosáhne tohoto nejmenšího
počtu odstraňování, protože při každém odstranění se v pozdějších úsecích
tratě zvýší rychlost o jedna.

Po odstranění zpomalováků buď nastane případ, kdy vlak jede na konci s
rychlostí jedna, nebo s vyšší rychlostí. V prvním případě máme vystaráno,
v druhém musíme od zadu odstranit $l - 1$ zrychlováků, kde $l$ je konečná
rychlost. Tento počet je zřejmě nejmenší. Zároveň jistě dostaneme validní
trať, protože při odstraňování odzadu nemůže dojít k propadnutí rychlosti
pod 1. 

Celý tento algoritmus má časovou asymptotickou složitost $\mathcal{O}(n)$.

\end{document}
