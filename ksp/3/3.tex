\documentclass{fkssolpub}

\usepackage[czech]{babel}
\usepackage{fontspec}
\usepackage{fkssugar}
\usepackage{amsmath}
\usepackage{graphicx}

\author{Ondřej Sedláček}
\school{Gymnázium Oty Pavla} 
\series{3}
\problem{3} 

\begin{document} 

Nejdříve si musíme uvědomit, že čím více máme hromádek, tím kyselina
odstraní víc věcí, tedy ze začátku se vyplatí nejprve dělit hromady
a pak až používat kyselinu na zneškodnění všech hromad. A protože
počet použití kyseliny závisí jen na nejvyšší hromadě, která zůstala
po dělení hromad na menší hromady, jeden z možných algoritmů nalezajících
řešení je následující.

Pro každou hromadu o výšce $v_i$ zjistíme, kolik operací potřebujeme
k tomu, abychom rozdělili všechny hromádky na hromádky o výšce nejvýše
$v_i$ a pak tyto hromady zničili kyselinou. Jejich počet určíme pomocí
tohoto vzorce:

\[
  \left\lceil \frac{v_1}{v_i} - 1 \right\rceil
    + \left\lceil \frac{v_2}{v_i} - 1 \right\rceil + ...
    + \left\lceil \frac{v_N}{v_i} - 1 \right\rceil + v_i
\]

Když tohle vyzkoušíme pro každou hromadu a najdeme minimum, nalezneme
optimální řešení v čase $\mathcal{O}(N^2)$, protože vyhodnocení tohoto
vzorce trvá $\mathcal{O}(N)$ času.

\end{document}
