\documentclass{fkssolpub}

\usepackage[czech]{babel}
\usepackage{fontspec}
\usepackage{fkssugar}
\usepackage{amsmath}

\author{Ondřej Sedláček}
\school{Gymnázium Oty Pavla} 
\series{36-Z1}
\problem{4} 

\begin{document}

Protože zajímavost každého předmětu je nezáporná, můžeme použít hladový
algoritmus. Ten bude fungovat tak, že v každém
kroku posouvá index posledního pytlíku, přepočítá si hmotnost a zajímavost
aktuálního úseku, pokud je hmotnost příliš vysoká, posunu první index a
přepočítám hmotnost a zajímavost, a následně zajímavost porovnám z aktuálním
maximem a případně si uložím indexy. Pokud přepočítávání v každém kroku
implementuje přepočítáváním z úseku v předchozím kroku (zjistíme, jak se změnila
při změně indexu hmotnost a zajímavost) nebo pomocí prefixových součtů, má
tento algoritmus časovou složitost $\mathcal{O}(n)$ a prostorovou složitost
$\mathcal{O}(1)$.

\end{document}
