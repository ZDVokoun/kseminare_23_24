\documentclass{fkssolpub}

\usepackage[czech]{babel}
\usepackage{fontspec}
\usepackage{fkssugar}
\usepackage{amsmath}
\usepackage{graphicx}

\author{Ondřej Sedláček}
\school{Gymnázium Oty Pavla} 
\series{2}
\problem{2} 

\begin{document} 

Jako součást mého algoritmu nejprve inicializujeme dvojrozměrné
pole celých čísel o velikosti $N \times N$ na nulu. Postupně pro
každou šišku projdeme všechny místa, které budou šiškou zasažena,
a přičteme k nim v tabulce jedničku. Tím získáme tabulku, kde pro
každé políčko máme, kolikrát bude zasažen. To bude trvat celkem
$\mathcal{O}(S \cdot N^2)$.

Následně pro určení, jaké mravence zachraňovat, provedeme na políčkách,
kde jsou mravenci, bucket sort podle počtu zasažení. Tím pak jsme
schopni určit, jaké mravence budeme zachraňovat tím, že budeme procházet
setřízené mravence od nejméně zasahovaného a budeme kontrolovat, jestli
počet potřebných krabiček nepřesáhl $K$. Tohle celé bude trvat
$\mathcal{O}(N^2 + S)$ času.

Celkem algoritmus bude tedy trvat $\mathcal{O}(S \cdot N^2)$ času.
Prostorová složitost je $\mathcal{O}(N^2 + S)$.

\end{document}
