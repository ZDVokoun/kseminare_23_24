\documentclass{fkssolpub}

\usepackage[czech]{babel}
\usepackage{fontspec}
\usepackage{fkssugar}
\usepackage{amsmath}

\author{Ondřej Sedláček}
\school{Gymnázium Oty Pavla} 
\series{36-4}
\problem{3} 

\begin{document}

Nechť je počet beden $k$ a počet kroků $n$.

Jedno přímočaré řešení následující. Nejprve si souřadnice krabic uložíme do
vyhledávacího pole a pak zkontrolujeme, jestli vozík vjel na pozici
nějaké krabice. Pokud ne, pokračujeme s dalším krokem, jinak aktualizujeme
polohu krabice, zkontrolujeme, jestli nebyla vytlačena na pozici jiné krabice,
a tohle budeme opakovat i s dalšími krabicemi, dokud jedna z krabic
nebude vytlačena na volné políčko. Pak se ale nejhorší případ stane
ve chvíli, kdy v každém kroku bude vozík tlačit stejné krabice, proto
je časová složitost $\mathcal{O}(n k \log k)$.

U předchozího algoritmu je jasné, že dost zpomalují právě případy, kdy
dochází k posouvání několika krabic najednou. Protože jsou však krabice
nerozlišitelné, můžeme si místo souřadnic všech krabic pamatovat všechny
řady krabic po horizontální a vertikální ose. Můžeme si tedy pomoci tím, že
si vytvoříme dva vyhledávací stromy ukládající intervaly řad, a to jednu řady
v horizontálním směru a jednu ve vertikálním směru, což bude stát $\mathcal{O}(k)$
času.

Teď si rozmyslíme několik operací, které chceme schopni být dělat. Posun jedné
celé řady zvládneme v $\mathcal{O}(\log k)$ čase, najdeme řadu obsahující krabici s danými
souřadnicemi a pak aktualizujeme interval nebo případně sloučíme vedlejší řadou.
Rozdělení intervalu zvládneme podobně v $\mathcal{O}(\log k)$ čase, znova najdeme
danou řadu a teď ji smažeme a místo ní přidáme dva rozdělené. Přidání nové
řady či rozšíření jiné je zřejmě taky v $\mathcal{O}(\log k)$.

Víme, že v každém kroku se můžou stát jen ty tři vyjmenované operace výše. Při kroku
v horizontálním směru se posune jedna celá řada v horizontálním směru, ve vertikálním
směru smažeme jednu krabici z nějaké řady a pak ji znova přidáme do jiné či vytvoříme
novou řadu. Proto v každém kroku provedeme $\mathcal{O}(\log k)$ operací, tedy
celková časová složitost je $\mathcal{O}(n \log k + k)$.

\end{document}
