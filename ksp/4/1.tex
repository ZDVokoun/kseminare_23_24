\documentclass{fkssolpub}

\usepackage[czech]{babel}
\usepackage{fontspec}
\usepackage{fkssugar}
\usepackage{amsmath}
\usepackage{graphicx}

\author{Ondřej Sedláček}
\school{Gymnázium Oty Pavla} 
\series{36-4}
\problem{1} 

\begin{document} 

Ze zadání víme, že zadaný graf je DAG. Dále si potřebujeme uvědomit,
že úkoly, kde jakkoli malé zpoždění způsobí zpoždění celé práce, jsou
ty úkoly, které se nachází na té nejdelší cestě v celém grafu, proto
naším úkolem je najít nejdelší cestu v DAG. To je známý problém,
pro který lze najít řešení v lineárním čase.

Jako první celý graf topologicky seřadíme. Pak postupně v tomto
řazení aktualizujeme každému synu aktuálně zpracovávanému vrcholu
nejdelší vzdálenost od kořene a odkaz na vrchol, který v nejdelší cestě
do tohoto vrcholu předcházel, což nám umožní zrekonstruovat celou cestu v lineárním čase.
Když pak najdeme vrcholy, který jsou nejdál od kořene, stačí z těchto vrcholů
zrekonstruovat nejdelší cestu a vrátit vrcholy, které na ní leží.

Celý tento algoritmus běží v čase $\mathcal{O}(n)$ a má prostorovou
složitost $\mathcal{O}(n)$.

\end{document}
