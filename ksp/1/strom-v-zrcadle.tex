\documentclass{fkssolpub}

\usepackage[czech]{babel}
\usepackage{fontspec}
\usepackage{fkssugar}
\usepackage{amsmath}

\author{Ondřej Sedláček}
\school{Gymnázium Oty Pavla} 
\series{1}
\problem{1} 

\begin{document}

Abychom zjistili, jestli je zadaný strom symetrický, musíme zkontrolovat, jestli
strom sestává ze dvou podstromů připojených na kořen a následně tyto podstromy
naráz procházet a kontrolovat (slovem naráz nemyslím paralelně), jestli navštívené vrcholy jsou stejné.

K tomu však musíme přijít na způsob, jak procházet binární stromy tak, abychom
používali jen konstantně paměti. Všechny vrcholy můžeme však navštívit
následujícím způsobem:

\begin{enumerate}
	\item Nejprve půjdeme do listu, který je nejvíce nalevo a který je z aktuálního
	      vrcholu dostupný průchodem dolů, neboli půjdeme vždy po
	      levým synovi, a pokud levý syn aktuálního vrcholu neexistuje, půjdeme po
	      pravém synovi.
	\item Následně si uložíme ukazatel a vrátíme se k otci. Při návštěvě otce
	      zkontrolujeme, jestli jsme se vrátili z pravého syna porovnáním ukazatelů.
	      Pokud ano, půjdeme k otci aktuálního vrcholu a postup opakujeme, jinak
	      půjdeme po pravém synovi a následně provedeme krok (1).
	\item Pokud chceme se vracet nahoru, ale ukazatel na otce je nulový,
	      prošli jsme celý graf a ukončíme běh algoritmu.
\end{enumerate}

Pro průchod levého podstromu použijeme tento postup, pro pravý podstrom prohodíme
levou a pravou stranu. Zároveň při průchodu dolů vždy zkontrolujeme, jestli syni
aktuálně procházených vrcholů jsou k sobě symetričtí a následně porovnáme hodnoty
aktuálních vrcholů.

Protože každý vrchol navštívíme nejvýše třikrát a při návštěvě vrcholu provedeme
konstantně operací, časová složitost tohoto algoritmu je $\mathcal{O}(n)$.

\end{document}
