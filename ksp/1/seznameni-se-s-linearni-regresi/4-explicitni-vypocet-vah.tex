\documentclass{fkssolpub}

\usepackage[czech]{babel}
\usepackage{fontspec}
\usepackage{fkssugar}
\usepackage{amsmath,amsfonts, amssymb}

\author{Ondřej Sedláček}
\school{Gymnázium Oty Pavla} 
\series{1}
\problem{S} 

\begin{document} 

\section{Úkol 4 -- Explicitní výpočet vah}

Nejdříve musíme určit vzorec, jak získat predikce pro každé dato jako vektor $\mathbf{p}$.
Pokud máme sloupcový vektor vah $\mathbf{w}$ a řádkový vektor featur $k$-tého
data $\mathbf{x_k}$, predikce pro $k$-té dato bude $p_k = \mathbf{x}_k \mathbf{w}$.
Pokud všechna data spojíme do jedné matice $\mathbf{X}$, kde každý řádek bude jedno dato,
dostaneme potřebný vzorec:

\[
  \mathbf{p} = \mathbf{X} \mathbf{w}
\]

Teď se přesuneme k soustavě rovnic, kterou jsme získali derivací MSE vůči vektoru
$\mathbf{w}$. Zde si ale musíme všimnout, že v každé rovnici provádíme skalární
součin s vektorem, kde jsou hodnoty $k$-té featury všech dat v trénovacím datasetu. 
Z toho vyplývá, že vektory $\mathbf{p}$ a $\mathbf{t}$ transformujeme maticí 
$\mathbf{X}^{\top}$, proto tuto soustavu rovnic zapíšeme maticově jako:

\[
  \mathbf{X}^{\top} \mathbf{p} = \mathbf{X}^\top \mathbf{t}
\]

Když dosadíme vzorec pro $\mathbf{p}$, dostaneme:

\[
  \mathbf{X}^\top \mathbf{X} \mathbf{w} = \mathbf{X}^\top \mathbf{t}
\]

\[
  \mathbf{w} = (\mathbf{X}^\top \mathbf{X})^{-1} \mathbf{X}^\top \mathbf{t}
\]

Což je explicitní vzorec pro výpočet optimálních vah.

\end{document}
