\documentclass{fkssolpub}

\usepackage[czech]{babel}
\usepackage{fontspec}
\usepackage{fkssugar}
\usepackage{amsmath}

\author{Ondřej Sedláček}
\school{Gymnázium Oty Pavla} 
\series{1}
\problem{5} 

\begin{document} 

Instrukci, kterou já navrhuji k přidání do nejlepšího programovacího jazyka, je instrukce
tetrace, neboli operaci, kdy $n$-krát umocníme číslo $m$ číslem $m$. Jeho formální zápis
je buď $^n m$ nebo $m \uparrow \uparrow n$. 

Pro implementaci by se pak hodilo mít k dispozici instrukci pro umocňování, i když
to není nutné. Tato instrukce by se buď

\end{document}
