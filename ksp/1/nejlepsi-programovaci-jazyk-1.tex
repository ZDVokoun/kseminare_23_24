\documentclass{fkssolpub}

\usepackage[czech]{babel}
\usepackage{fontspec}
\usepackage{fkssugar}
\usepackage{amsmath}

\author{Ondřej Sedláček}
\school{Gymnázium Oty Pavla} 
\series{1}
\problem{5} 

\begin{document} 

Instrukci, kterou já navrhuji k přidání do nejlepšího programovacího jazyka, je instrukce
tetrace, neboli operaci, kdy $n$-krát umocníme číslo $m$ číslem $m$. Jeho matematický zápis
je většinou $m \mathbin{\uparrow \uparrow} n$, proto bych tuto operaci zapisoval jako 
\verb|^^|. Čísla $m$ a $n$ by se pak získala ze zásobníku. V implementaci, pokud bude k 
dispozici, by se použila instrukce umocňování, jinak by se implementoval pomocí prostého 
násobení.

Asi mě můžete podepsat jako autora instrukce, je mi to popravdě celkem jedno.

\end{document}
